
%----------------------------------------------------------------------------------------
%	SECTION 6
%----------------------------------------------------------------------------------------
\newpage
\section{Sweco, August 2018}

%\subsection{Environmental Building Certification Summer Placement at Sweco in Malm\"o, Sweden}

\begin{wrapfigure}{r}{0.2\textwidth}
	\centering
	\begin{subfigure}{0.2\textwidth}
		\centering
		\includegraphics[width=\textwidth]{figures/knapp-mb.jpg}
		%          \rule{\textwidth}{0.5pt} % use line???
		%\caption{Latent heat storage}
		%\label{fig:latentheatstorage}
	\end{subfigure}
	\begin{subfigure}{0.2\textwidth}
		\centering
		\includegraphics[width=\textwidth]{figures/sgbc.png}
		%          \rule{\textwidth}{0.5pt} % use line???
		%\caption{Latent heat emission}
		%\label{fig:latentheatemission}
	\end{subfigure}
	%\rule{0.2\textwidth}{0.5pt} % use line???
	%\caption[Load shifting, an example of DSM.]{Load shifting, an example of DSM \citep{USAID}).}
	\label{fig:mb}
\end{wrapfigure}


I really enjoyed my internship at Sweco in spite of its brevity (only three weeks).
I worked in environmental building certification alongside my supervisor who was the only certifier in the Malm\"o office in Sweden.
The certification system we used was \textit{Milj\"obyggnad} by the Sweden Green Building Council.
Upon reading about \textit{Milj\"obyggnad}, I was immediately drawing parallels to and comparing it to BREEAM (Building Research Establishment Environmental Assessment Method), which I had learned about in \textit{Sustainable and Intelligent Buildings}.
Unlike Sunamp, this more technical work was more along my lines of interest and more relevant to the education I have received at Heriot-Watt University.

I helped my supervisor carry out pilot studies for 11 existing properties.
The aim was to assess the properties' current state in relation to the \textit{Milj\"obyggnad} criteria and list the necessary refurbishment measures that would most likely get the buildings certified.
This was done to help the real estate owners decide whether it was worth to go ahead with the refurbishments to achieve certification.
Due to my limited time at Sweco, I only managed to assist my supervisor with the assessment of one property which consisted of three different buildings.
I analysed, amongst other things, solar heat gains, energy consumption, thermal comfort and daylighting in accordance with the \textit{Milj\"obyggnad} methods and the Swedish building regulations (\textit{Boverkets Byggregler (BBR)}).
I have not included extracts of my work due to confidentiality reasons.

%For the second year in a row, my application to Skanska had been rejected.
%I tried to use my dad's contacts to get work, but these people never got back to me, despite my multiple attempts to reach them.
%Sweco did not have an official summer placement application process, so I called the relevant regional head of Sweco's building services department.
%He seemed interested to hear about my previous work experience, especially at Arup.
%He contacted somebody else in the Malmo office, who invited me to interview.
%My persistence to reach out to and politely remind the people of Sweco of my interest helped me obtain an offer to work with them.

Despite being Swedish, I have never gone to a school where the primary language of instruction was Swedish.
I am most confident in the English language and was aware from the start that my level in Swedish might pose a problem, especially with construction-related jargon (which I have learned in English at Heriot-Watt University).
I made my supervisor aware of this from the start.
I found my general Swedish level improving daily.
Learning the Swedish jargon took a bit of time, as I questioned and looked up words.
I found myself becoming more comfortable with the Swedish jargon during my brief three week placement at Sweco.
I believe with some more time I would eventually get the hang of it.
However, the work required some level of report writing in Swedish as well, which I was not entirely comfortable with.
I got through this through some autonomous exploration of how to phrase sentences and a couple of questions to my supervisor.
Either way, I have decided that, if I do end up working in Sweden, I will take a course to improve my Swedish.
This would improve my confidence to write reports and converse with colleagues in the workplace.

Other than that, my supervisor seemed really pleased with the work I had done, so much so that I was actually offered a job after graduation.
%That is another thing I will need to learn about if I work in Sweden: their building regulations.
%Although, that is something I would need to learn working anywhere, even the UK.