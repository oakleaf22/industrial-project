%----------------------------------------------------------------------------------------
%	SECTION 3
%----------------------------------------------------------------------------------------

\section{Design (D)}

\subsection*{D1(i, -)}

Multiple courses have developed and contributed to my understanding and ability to evaluate business, customer and user needs:
\textit{Introduction to Design},
\textit{Introduction to the Environment},
\textit{Environment and Behaviour},
\textit{Critical Architectural Studies},
\textit{Electrical and Lighting Services for Buildings},
\textit{Procurement and Contracts},
\textit{Facilities Management Principles},
\textit{Inclusive and Safe Environments},
and \textit{Design Project}.
The concept of a good brief that reflects the business, customer and user needs has been emphasised throughout the AE programme.
I have been able to evaluate such needs through various assignments and in different contexts, such as technical aspects (air quality, lighting levels and fire safety), users with different kinds of disabilities, and public perception.
During my placement at Hoare Lea, I was able to evaluate the design brief of a mixed-use new-build for needs regarding fire safety in order to design the fire detection and alarm systems.

%Some of these courses considered more technical needs such as good air quality and lighting levels, which are dependent on the functions of spaces.
%In \textit{Procurement and Contracts} and \textit{Facilities Management Principles}, I learned about the different stakeholders involved in a project and their respective needs and interests.


\subsection*{D2(i, -)}

\hl{Investigating and defining problems...}


\subsection*{D3(i, b, m)}

Working with information that may be incomplete or uncertain is something that I have found difficult.
\textit{Design Project B}, \textit{Design Project} and \textit{Laboratory Project} are courses that particularly challenged me in this area and helped me develop the skill.
During these projects I learned to make assumptions and use iterative/ trial-and-error processes to come up with a satisfactory design solution.
This was done, for example, for the sizing of pipes in the design projects.
\hl{However, I do not think I have ever gone as far as to quantify the effect of incomplete or uncertain information on a design and to use research to mitigate the deficiencies.}

\hl{See notes on Notion print-out from 26 Oct 2018}


\subsection*{D4(i, -)}

All design projects have developed my skills in creating design solutions that are fit for purpose:
\textit{1st year collaborative project},
\textit{Design Project A},
\textit{Design Project B},
\textit{Critical Architectural Studies},
\textit{Energy and Buildings},
and \textit{Design Project}, including the \textit{4th year collaborative project}.
\hl{Give examples?!
However, I cannot think of an example where the whole lifecycle was considered, especially disposal.}


\subsection*{D5(i, -)}

\hl{Plan and manage the design process, including cost drivers...
Design Project? I didn't do the timeline or costs...
I have always kind of avoided doing costing (CAS, collab projects (for appropriate reason)).}

\hl{See notes on Notion print-out from 26 Oct 2018}


\subsection*{D6}

\hl{Just in the context of design?}

Most of the assignments I have written and presentations I have given contributed to the development of my skill of communicating to technical or non-technical audiences.
Courses and work placements that have contributed to this skill development include:
\textit{Design Project A},
\textit{Construction Technology 2},
\textit{Hydraulics and Hydrology A},
\textit{Environment and Behaviour},
\textit{Design Project B},
\textit{Critical Architectural Studies},
\textit{Thermal Performance Studies},
\textit{Design Issues},
\textit{Facilities Management Principles},
\textit{Inclusive and Safe Environments},
\textit{Laboratory Project},
\textit{Design Project},
\textit{Dissertation},
\textit{Climate Change, Sustainability and Adaptation},
Sunamp,
Sweco?
Instances of communicating to non-technical audiences include my POSTnote assignment for \textit{Climate Change, Sustainability and Adaptation}, where I wrote about a technical topic to members of the UK parliament, and the Product Information Sheets that I created for Sunamp, which needed to convey the technical operation of their products in layman's terms.
Most laboratory reports I have written, e.g. for \textit{Laboratory Project}, were of a technical nature.


\subsection*{D7m}

I have been exposed to various UK building design and construction processes (especially the Royal Institute of British Architects (RIBA) Plan of Work 2013) and procurement routes during the following courses and work placements:
Arup,
\textit{Procurement and Contracts},
\textit{Facilities Management Principles},
Hoare Lea,
and
\textit{Dissertation}.
I have been able to contrast this knowledge with my (albeit limited) knowledge of Swedish building design and construction processes gained during my placement at Hultin \& Lundquist.
The thermofluids part of \textit{Laboratory Project} taught me about the methodology for optimising a design.
This knowledge has been complemented by information about specific \hl{parts/ scopes} of the design process, such as the briefing stage, site analysis, POEs and life cycle assessments, learned in
\textit{Construction Technology 1},
\textit{Introduction to Design},
\textit{Environment and Behaviour},
\textit{Facilities Management Principles},
and \textit{Sustainable and Intelligent Buildings}.
I have also delved into a study of BIM, a new way to approach collaborative building design which is the new ``Holy Grail" of design processes, during my \textit{Dissertation} and \textit{Innovation in Construction Practice}.
%UK vs Sweden: Arup, 
%HL, FMP, DST, P\&C vs H\&L
%BIM: DST, ICP
%Engineering optimisation: LAB
%From site analysis to POEs + LCA: Con tech 1, Intro to Design, Env and Beh, SIB
All of these courses and experiences have contributed to my wide knowledge and comprehensive understanding of design processes and methodologies.

I have developed my ability to apply and adapt this wide repository of knowledge and understanding of design processes and methodologies in the unfamiliar situations presented in \textit{Design Project} and \textit{Critical Architectural Studies}.
\hl{Elaborate?}
%Application in unfamiliar situations: DP, CAS


\subsection*{D8m}

\hl{New needs?}

\hl{See notes on Notion print-out from 26 Oct 2018}


