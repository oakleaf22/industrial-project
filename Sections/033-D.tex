%----------------------------------------------------------------------------------------
%	SECTION 3
%----------------------------------------------------------------------------------------

\section{Design (D)}

\subsection*{D1(i, -)}

\begin{wraptable}{r}{0.2\textwidth}
	\begin{tabular}{|ll|}
		\hline
		\multicolumn{2}{|c|}{\cellcolor[HTML]{F8A102}\textbf{D1(i, -) \master}} \\ \hline
		\ID & \IE \\
		\EnvBeh & \CAS \\
		\ELS & \PC \\
		\FMP & \PRJ \\
		\ISE & Sunamp \\ \hline
	\end{tabular}
\end{wraptable}

Multiple courses have developed and contributed to my understanding and ability to evaluate business, customer and user needs.
The concept of a good brief that reflects the business, customer and user needs has been emphasised throughout the AE programme.
I have been able to evaluate such needs through various assignments and in different contexts, such as technical aspects (air quality, lighting levels and fire safety), users with different kinds of disabilities, and public perception.
During my placement at Hoare Lea, I was able to evaluate the design brief of a mixed-use new-build for needs regarding fire safety in order to design the fire detection and alarm systems.
Moreover, half of the marks for \CASTitle \space were based on a group project and our ability to respond to the design brief of Newington Library.
I cannot find the mark I got specifically for this project, but our group won first prize in sustainability (see Figure~\ref{fig_award}) and I got an overall A for that course.
I can thus deduce that I achieved LOs D1i and D1.

%Some of these courses considered more technical needs such as good air quality and lighting levels, which are dependent on the functions of spaces.
%In \textit{Procurement and Contracts} and \textit{Facilities Management Principles}, I learned about the different stakeholders involved in a project and their respective needs and interests.







\subsection*{D2(i, -)}

\begin{wraptable}{r}{0.2\textwidth}
	\begin{tabular}{|ll|}
		\hline
		\multicolumn{2}{|c|}{\cellcolor[HTML]{F8A102}\textbf{D2(i, -) \master}} \\ \hline
		\DI & \DST \\
		\LAB & \CCSA \\ \hline
	\end{tabular}
\end{wraptable}

I have developed the ability to define and investigate a problem in my later years at Heriot-Watt University.
For example, for \DITitle, we were required to write a short report on the impact of climate on an area of construction/ building services in a country of our own choice.
For this, I selected Guyana, a tropical country in South America.
A Guyanese local told me about the lack of shading on buildings in their country.
This was my starting point, from where I defined and investigated the problem.
I conducted a literature review that revealed the need for shading in tropical countries to help avoid overheating in buildings and reduce cooling energy loads.
Hence, a lack of shading in a tropical climate is problematic as overheating poses health risks to building occupants (e.g  heat stress and even death) and the increased use of air conditioners is not only expensive to run, but emits greenhouse gases, thus contributing to climate change.
I then proceeded to investigate the extent of the shading deficiency on Guyana's commercial buildings by conducting an evidence review.
This included examining Guyanese newspaper articles and speaking with a couple people from Guyana, one of whom is a local architect.
I was able to confirm the increasing lack of shading on modern commercial buildings and identify the reasons for this.
These include a lack of regulations on buildings and designers, and the appeal of the international glass-dominated architecture style (enabled due to technologies like air conditioning).

I have similarly defined and investigated problems for the \DSTTitle \space and in assignments for \LABTitle \space and \CCSATitle.
My marks for these assignments (one B and four As) demonstrate my full achievement of LOs D2i and D2.








\subsection*{D3(i, b, m)}

\begin{wraptable}{r}{0.2\textwidth}
	\begin{tabular}{|ll|}
		\hline
		\multicolumn{2}{|c|}{\cellcolor[HTML]{F8A102}\textbf{D3(i, b, m) \littlemaster}} \\ \hline
		\DPB & \PRJ \\
		\LAB &  \\
		Sunamp & Sweco \\ \hline
	\end{tabular}
\end{wraptable}

Working with information that may be incomplete or uncertain is something that I have found difficult.
\textit{Design Project B}, \textit{Design Project} and \textit{Laboratory Project} are courses that particularly challenged me in this area and helped me develop the skill.
During these projects I learned to make assumptions and use iterative/ trial-and-error processes to come up with a satisfactory design solution.
The fact that I passed all these courses demonstrates that I have some ability to work with incomplete or uncertain information.
%This was done, for example, for the sizing of pipes in the design projects.
However, I still consider my ability to do this to be rather weak.

I do not think I have ever gone as far as to quantify the effect of incomplete or uncertain information on a design and to use research to mitigate the deficiencies.
I need to develop these abilities as I presume they will be necessary in industry, where the information required to carry out design tasks may not always be certain or complete.








\subsection*{D4(i, -)}

\begin{wraptable}{r}{0.2\textwidth}
	\begin{tabular}{|ll|}
		\hline
		\multicolumn{2}{|c|}{\cellcolor[HTML]{F8A102}\textbf{D4(i, -) \nomaster}} \\ \hline
		\DPA & \DPB \\
		\CAS & \EnBldgs \\
		\PRJ &  \\ \hline
	\end{tabular}
\end{wraptable}

All of my design projects at Heriot-Watt University have developed my skills in creating design solutions that are fit for purpose through application of problem-solving skills, technical knowledge and understanding.
%\textit{1st year collaborative project},
%\textit{Design Project A},
%\textit{Design Project B},
%\textit{Critical Architectural Studies},
%\textit{Energy and Buildings},
%and \textit{Design Project}, including the \textit{4th year collaborative project}.
%\hl{Give examples?! However, I cannot think of an example where the whole life cycle was considered, especially disposal.}
For example, in \CASTitle, I ensured the lighting design of Newington Library was fit for purpose by helping determine the building orientation and the layout of the internal rooms, considering the possible overshadowing effect of neighbouring buildings and maximising daylighting.
My customised natural lighting design was commended in the design competition (see Figure~\ref{fig_award}).

I, however, have not established rigorous design solutions that are fit for purpose for \emph{all} aspects of the problem.
So far, I have only created bespoke solutions that considered one or some aspects of a building's life cycle, e.g. production, operation and/ or maintenance.
I cannot think of an example where I considered the demolition and disposal of a building.
Therefore, I have not accomplished LO D4.
Perhaps I will get the opportunity to create a `life cycle solution' in next semester's course \LCBTitle.









\subsection*{D5(i, -)}

\begin{wraptable}{r}{0.2\textwidth}
	\begin{tabular}{|ll|}
		\hline
		\multicolumn{2}{|c|}{\cellcolor[HTML]{F8A102}\textbf{D5(i, -)} \nomaster} \\ \hline
		\IE & \DPA \\
		\DPB & \CAS \\
		\PRJ & \DST \\
		\LAB & \CCSA \\ \hline
	\end{tabular}
\end{wraptable}

I have had some experience in planning and managing design processes, notably in group design projects (whether it was the design of a building or a poster \ldots).
I have also evaluated the outcomes of a design process, notably in a reflective report submitted after the design of a library in \CASTitle.
Some reflective points from that report (on which I got an A$+$) on what makes a good design process are \citep{eklowCAS}:
\begin{itemize}
	\item ``a strong concept [\ldots] helped guide us through all of our decision-making and eventually come up with a good-quality design"
	\item ``The final group design could have been improved [\ldots] if the building solution had been finalised earlier [\ldots]. This could have been achieved by planning the design timeline right from the start, something we failed to do."
	\item ``I learned that tasks should be sized and assigned according to each individual’s commitment/ reliability" rather than in fairness.
\end{itemize}

I have, however, never planned or managed the cost drivers of a design process, despite having had opportunities to do this.
The task of costing was either allocated to another group member, or (as was the case for the \PRJTitle) I failed to do it due to poor time management.







\subsection*{D6}

\begin{wraptable}{r}{0.2\textwidth}
	\begin{tabular}{|ll|}
		\hline
		\multicolumn{2}{|c|}{\cellcolor[HTML]{F8A102}\textbf{D6 \master}} \\ \hline
		\DPA & \ConTechOne \\
		\HYD & \EnvBeh \\
		\DPB & \CAS \\
		\TPS & \DI \\
		\FMP & \PRJ \\
		\DST & \LAB \\
		\ISE & \CCSA \\
		Sunamp & Sweco \\ \hline
	\end{tabular}
\end{wraptable}

Most of the assignments I have written and presentations I have given contributed to the development of my skill of communicating to technical or non-technical audiences.
Instances of communicating to non-technical audiences include my POSTnote assignment for \textit{Climate Change, Sustainability and Adaptation}, where I wrote about a technical topic to members of the UK parliament (see more under the second bullet point in Section~\ref{sec:G1}), and the UniQ PISs that I created for Sunamp which needed to convey the technical operation of their products in layman's terms.
Most laboratory reports I have written, e.g. for \textit{Laboratory Project}, were of a technical nature.
To name a few, my provisional grade of B in the POSTnote and my final grades of A in the laboratory reports demonstrate my accomplishment of LO D6.







\subsection*{D7m}

\begin{wraptable}[9]{r}{0.2\textwidth}
	\begin{tabular}{|ll|}
		\hline
		\multicolumn{2}{|c|}{\cellcolor[HTML]{F8A102}\textbf{D7m \nomaster}} \\ \hline
		\ConTechOne & \ID \\
		\EnvBeh & \CAS \\
		\PC & \FMP \\
		\PRJ & \DST \\
		\LAB & \SIB \\
		\ICP & Arup \\
		H{\&}L & Hoare Lea \\ \hline
	\end{tabular}
\end{wraptable}

Multiple courses and work placements have contributed to my wide knowledge and comprehensive understanding of design processes and methodologies.
\PC, amongst other courses, has exposed me to various building design and construction procurement routes and processes, particularly the RIBA Plan of Work 2013.
The thermofluids part of \textit{Laboratory Project} taught me about the methodology for optimising a design.
This knowledge has been complemented by information about specific parts of the design process, such as the briefing stage, site analysis, POEs and life cycle assessments.
%UK vs Sweden: Arup, 
%HL, FMP, DST, P\&C vs H\&L
%BIM: DST, ICP
%Engineering optimisation: LAB
%From site analysis to POEs + LCA: Con tech 1, Intro to Design, Env and Beh, SIB
I have also delved into a study of BIM, a new way to approach collaborative building design which is the new ``Holy Grail" of design processes, during the \DSTTitle.
The fact that I received an overall A grade in the three aforementioned courses demonstrates my knowledge and understanding of design processes and methodologies.

Furthermore, I have developed an ability to apply and adapt these design processes and methodologies in unfamiliar situations.
For example, the \PRJTitle \space comprised of two stages which resembled Stages 2 (Concept Design) and 3 (Developed Design) of the RIBA Plan of Work 2013.
And in \CASTitle, I took the initiative to conduct a site analysis and develop a POE questionnaire to improve the design of the library.
Having passed both courses proves my ability to apply and adapt design processes and methodologies in unfamiliar situations.








\subsection*{D8m}

\begin{wraptable}{r}{0.2\textwidth}
	\begin{tabular}{|ll|}
		\hline
		\multicolumn{2}{|c|}{\cellcolor[HTML]{F8A102}\textbf{D8m} \nomaster} \\ \hline
		\PRJ & Sunamp \\ \hline
	\end{tabular}
\end{wraptable}

The way that I developed the UniQ Product Selection Quiz (described in Section~\ref{sec:quiz}) is a demonstration of my ability to generate an innovative design for products, systems, components or processes to fulfil new needs.
I also came up with the innovative idea of generating tidal power for the 4\textsuperscript{th} year collaborative project (described in Section~\ref{sec:P10m}), but this was not fully designed.
Since I have not created many innovative designs to fulfil new needs, I would not say that I have fully developed or mastered this skill.


