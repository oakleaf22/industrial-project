%----------------------------------------------------------------------------------------
%	SECTION 3
%----------------------------------------------------------------------------------------

\section{Design (D)}

\subsection*{D1(i, -)}

\begin{wraptable}{r}{0.2\textwidth}
	\begin{tabular}{|ll|}
		\hline
		\multicolumn{2}{|c|}{\cellcolor[HTML]{F8A102}\textbf{D1(i, -) \master}} \\ \hline
		\ID & \IE \\
		\EnvBeh & \CAS \\
		\ELS & \PC \\
		\FMP & \PRJ \\
		\ISE & Sunamp \\ \hline
	\end{tabular}
\end{wraptable}

Multiple courses have developed and contributed to my understanding and ability to evaluate business, customer and user needs.
The concept of a good brief that reflects the business, customer and user needs has been emphasised throughout the AE programme.
I have been able to evaluate such needs through various assignments and in different contexts, such as technical aspects (air quality, lighting levels and fire safety), users with different kinds of disabilities, and public perception.
During my placement at Hoare Lea, I was able to evaluate the design brief of a mixed-use new-build for needs regarding fire safety in order to design the fire detection and alarm systems.
Moreover, half of the marks for \CASTitle \space were based on a group project and our ability to respond to the design brief of Newington Library.
I cannot find the mark I got specifically for this project, but our group won first prize in sustainability (see Figure~\ref{fig_award}) and I got an overall A for that course.
I can thus deduce that I achieved LOs D1i and D1.

%Some of these courses considered more technical needs such as good air quality and lighting levels, which are dependent on the functions of spaces.
%In \textit{Procurement and Contracts} and \textit{Facilities Management Principles}, I learned about the different stakeholders involved in a project and their respective needs and interests.







\subsection*{D2(i, -)}

I have developed the ability to define and investigate a problem in my later years at Heriot-Watt University.
For example, for \DITitle, we were required to ...






\subsection*{D3(i, b, m)}

Working with information that may be incomplete or uncertain is something that I have found difficult.
\textit{Design Project B}, \textit{Design Project} and \textit{Laboratory Project} are courses that particularly challenged me in this area and helped me develop the skill.
During these projects I learned to make assumptions and use iterative/ trial-and-error processes to come up with a satisfactory design solution.
This was done, for example, for the sizing of pipes in the design projects.
\hl{However, I do not think I have ever gone as far as to quantify the effect of incomplete or uncertain information on a design and to use research to mitigate the deficiencies.}

\hl{See notes on Notion print-out from 26 Oct 2018}


\subsection*{D4(i, -)}

All design projects have developed my skills in creating design solutions that are fit for purpose:
\textit{1st year collaborative project},
\textit{Design Project A},
\textit{Design Project B},
\textit{Critical Architectural Studies},
\textit{Energy and Buildings},
and \textit{Design Project}, including the \textit{4th year collaborative project}.
\hl{Give examples?! However, I cannot think of an example where the whole lifecycle was considered, especially disposal.}




\subsection*{D5(i, -)}

\begin{wraptable}[5]{r}{0.2\textwidth}
	\begin{tabular}{|ll|}
		\hline
		\multicolumn{2}{|c|}{\cellcolor[HTML]{F8A102}\textbf{D5(i, -)}} \\ \hline
		\IE & \DPA \\
		\DPB & \CAS \\
		\PRJ & \DST \\
		\LAB & \CCSA \\ \hline
	\end{tabular}
\end{wraptable}

I have had some experience in planning and managing design processes, notably in group design projects (whether it was the design of a building or a poster \ldots).
I have also evaluated the outcomes of a design process, notably in a reflective report submitted after the design of a library in \CASTitle.
Some reflective points from that report (on which I got an A$+$) on what makes a good design process are \citep{eklowCAS}:
\begin{itemize}
	\item ``a strong concept [\ldots] helped guide us through all of our decision-making and eventually come up with a good-quality design"
	\item ``The final group design could have been improved [\ldots] if the building solution had been finalised earlier [\ldots]. This could have been achieved by planning the design timeline right from the start, something we failed to do."
	\item ``I learned that tasks should be sized and assigned according to each individual’s commitment/ reliability" rather than in fairness.
\end{itemize}

I have, however, never planned or managed the cost drivers of a design process, despite having had opportunities to do this.
The task of costing was either allocated to another group member, or (as was the case for the \PRJTitle) I failed to do it due to poor time management.







\subsection*{D6}

\hl{Just in the context of design?}

Most of the assignments I have written and presentations I have given contributed to the development of my skill of communicating to technical or non-technical audiences.
Courses and work placements that have contributed to this skill development include:
\textit{Design Project A},
\textit{Construction Technology 2},
\textit{Hydraulics and Hydrology A},
\textit{Environment and Behaviour},
\textit{Design Project B},
\textit{Critical Architectural Studies},
\textit{Thermal Performance Studies},
\textit{Design Issues},
\textit{Facilities Management Principles},
\textit{Inclusive and Safe Environments},
\textit{Laboratory Project},
\textit{Design Project},
\textit{Dissertation},
\textit{Climate Change, Sustainability and Adaptation},
Sunamp,
Sweco?
Instances of communicating to non-technical audiences include my POSTnote assignment for \textit{Climate Change, Sustainability and Adaptation}, where I wrote about a technical topic to members of the UK parliament, and the Product Information Sheets that I created for Sunamp, which needed to convey the technical operation of their products in layman's terms.
Most laboratory reports I have written, e.g. for \textit{Laboratory Project}, were of a technical nature.


\subsection*{D7m}

I have been exposed to various UK building design and construction processes (especially the RIBA Plan of Work 2013) and procurement routes during the following courses and work placements:
Arup,
\textit{Procurement and Contracts},
\textit{Facilities Management Principles},
Hoare Lea,
and
\textit{Dissertation}.
I have been able to contrast this knowledge with my (albeit limited) knowledge of Swedish building design and construction processes gained during my placement at Hultin \& Lundquist.
The thermofluids part of \textit{Laboratory Project} taught me about the methodology for optimising a design.
This knowledge has been complemented by information about specific \hl{parts/ scopes} of the design process, such as the briefing stage, site analysis, POEs and life cycle assessments, learned in
\textit{Construction Technology 1},
\textit{Introduction to Design},
\textit{Environment and Behaviour},
\textit{Facilities Management Principles},
and \textit{Sustainable and Intelligent Buildings}.
I have also delved into a study of BIM, a new way to approach collaborative building design which is the new ``Holy Grail" of design processes, during my \textit{Dissertation} and \textit{Innovation in Construction Practice}.
%UK vs Sweden: Arup, 
%HL, FMP, DST, P\&C vs H\&L
%BIM: DST, ICP
%Engineering optimisation: LAB
%From site analysis to POEs + LCA: Con tech 1, Intro to Design, Env and Beh, SIB
All of these courses and experiences have contributed to my wide knowledge and comprehensive understanding of design processes and methodologies.

I have developed my ability to apply and adapt this wide repository of knowledge and understanding of design processes and methodologies in the unfamiliar situations presented in \textit{Design Project} and \textit{Critical Architectural Studies}.
\hl{Elaborate?}
%Application in unfamiliar situations: DP, CAS


\subsection*{D8m}

\begin{wraptable}{r}{0.2\textwidth}
	\begin{tabular}{|ll|}
		\hline
		\multicolumn{2}{|c|}{\cellcolor[HTML]{F8A102}\textbf{D8m}} \\ \hline
		\PRJ & Sunamp \\ \hline
	\end{tabular}
\end{wraptable}

The way that I developed the UniQ Product Selection Quiz (described in Section~\ref{sec:quiz}) is a demonstration of my ability to generate an innovative design for products, systems, components or processes to fulfil new needs.
I also came up with the innovative idea of generating tidal power for my 4\textsuperscript{th} year collaborative project (described in Section~\ref{sec:P10m}), but this was not fully designed.
Since I have not created many innovative designs to fulfil new needs, I would not say that I have fully developed or mastered this skill.


