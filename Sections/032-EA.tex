%----------------------------------------------------------------------------------------
%	SECTION 2
%----------------------------------------------------------------------------------------

\section{Engineering Analysis (EA)} \label{EA}

\subsection*{EA1(i, b, m)}

Conducting laboratory experiments and modelling exercises in \textit{Energy Principles and Applications}, \textit{Design Software Applications}, \textit{Thermal Performance Studies} and \textit{Laboratory Project} further developed my skills in monitoring, interpreting and applying the results of analysis to bring about continuous improvement (EA1i).
\textit{Laboratory Project} went a step further and taught me the principles and process to engineering a better solution (EA1b and EA1m).
These principles were applied in an engineering project which aimed to optimise the design of a cooling coil.
\hl{Analysis of key engineering processes???}





\subsection*{EA2(i, -)}

A range of courses have developed my ability to understand and explain the performance of systems and components through the use of quantitative/ analytical methods and/ or modelling techniques:
\textit{Construction Technology 2},
\textit{Acoustics and Architectural Design},
\textit{Hydraulics and Hydrology A},
\textit{Energy Principles and Applications},
\textit{Thermal Performance Studies},
\textit{Laboratory Project},
and \textit{Design Project}.
For example, in \textit{Hydraulics and Hydrology A}, I learned how to use quantitative methods to distinguish laminar flows from turbulent flows.
And when we were studying thermofluids in \textit{Laboratory Project}, I used both CFD modelling and analytical methods to understand and describe the performance of a cooling coil in an air conditioning unit.

I have also applied quantitative/ analytical methods and/ or modelling techniques to classify and describe the performance of systems during my work placements at Arup and Sweco.
In particular, at Sweco I used a combination of the aforementioned methods and techniques to describe the performance of some buildings in terms of solar heat gain, energy consumption, thermal comfort in winter and summer, and daylighting.
The results of these analyses were then classified according to the criteria of \textit{Miljöbyggnad}, i.e. the Swedish environmental certification system for buildings: Gold, Silver, Bronze or Disapproved.





\subsection*{EA3(i, b, m)} \label{EA3}

I have developed my ability to use the results of analysis and apply quantitative and computational methods to solve engineering problems and subsequently recommend or implement appropriate action in the following courses:
\textit{Laboratory Project},
\hl{...}
For example, a combination of calculations, CFD modelling and a physical laboratory experiment were used to optimise the design of a cooling coil in \textit{Laboratory Project}.
This was an iterative process of solving problems and implementing appropriate actions to come up with a better solution.





\subsection*{EA4(i, b, m)}

\hl{If I look overll at all the steps (integrated), I can go down that path. but if I just look at the step that isn't working, I'd go down another path. Look at something that wasn't working, and . CCSA eposter: not to miss point of question/ topic (how to harness lessons from past climate events); focusing on other parts of question sets us up to fail. Because I know about this, I can apply it to engineering in the following ways. Sunamp PISs. Made sure whole process was more carefully thought out than it was when I was given the job. I had an overview of the whole thing. I made sure that the document enabled people to use the batteries as efficiently as possible. Can't do that by looking at each point of its own - it's got to flow as a document. specific application of heat batteries  - know-how  of technology and application (0.5 page)}



\subsection*{EA5m}

In addition to my work placements at Hoare Lea and Sunamp, courses that I have developed my ability to use fundamental knowledge to investigate new and emerging technologies are
\textit{Energy and Buildings},
\textit{Innovation in Construction Practice},
\textit{Design Project},
and \textit{Dissertation}.
My most extensive investigation must be of Sunamp's heat batteries, having written the information sheets for their UniQ product range.





\subsection*{EA6m}

\hl{The only example I can think of is when I tried to calculate the yield of tidal power. This was an unfamiliar problem and I could only find information online on how to calculate yields from hydroelectric turbines (?). Thus, I had to use a combination of the information I found, assumptions and my own fundamental mathematical skillset. Still, I never showed my workings to anyone, so I can't be sure if they were correct. Fan's physics/ mathematical exercises at start of TPS? Link to EL3?}

\hl{See notes on Notion print-out from 26 Oct 2018}


