%----------------------------------------------------------------------------------------
%	SECTION 4
%----------------------------------------------------------------------------------------

\section{Economic, Legal, Social, Ethical and Environmental Context (EL)}

\subsection*{EL1(-, m)}

I gained an understanding of the need for a high level of professional and ethical conduct in engineering and a knowledge of professional codes of conduct in \textit{Procurement and Contracts}.
I familiarised myself with CIBSE's and \hl{...} professional codes of conduct in an assignment where \hl{...}
This course and my ethical induction at Arup also taught me how ethical dilemmas can arise.
\hl{Elaborate?}


\subsection*{EL2}


\subsection*{EL3(i, b, m)}

Gantt chart? Learned about in ICP.
DP (but I didn't do this).
BIM process?

\hl{See notes on Notion print-out from 26 Oct 2018}


\subsection*{EL4(i, -)}

According to the oft-quoted Brundtland report, sustainable development is an approach to progress which meets the needs of the present generation without compromising the ability of future generations to meet their own needs.
\hl{Can't think of examples. Also, ability to apply quantitative techniques???}



\subsection*{EL5(-, m)}

The following courses increased my awareness of legal requirements governing engineering activities including \hl{health and safety (H\&S)}, liability issues, contracts and accessibility:
\textit{Design Issues},
\textit{Procurement and Contracts},
and \textit{Inclusive and Safe Environments}.
In particular, in \textit{Design Issues} I learned about the various UK regulations surrounding H\&S, including the Health and Safety at Work Act 1974 and the Construction (Design and Management) Regulations 2015, a.k.a. the CDM Regulations.
The CDM Regulations emphasise that H\&S need to be considered, not just on site, but throughout the planning and design stages also.
The designers, for example, have a responsibility to assess and eliminate or mitigate the H\&S risks associated with their design, whether that is during the construction or operation of the building.
Furthermore, this course broadened my awareness of legal requirements outside of the UK.
I learned about the H\&S regulations in France, the lack of building regulations in Guyana, and the ambitious building regulations with regards to sustainability in Singapore.
\hl{Is that a good demonstration of my knowledge? David: provide two references.}

I have also applied my knowledge of legal requirements in a project for \textit{Inclusive and Safe Environments} and during my placements at Hoare Lea and Sweco.
At Hoare Lea I produced \hl{RIBA} Stage 3 layouts of fire detection and alarm installations for a new-build.
These were made to comply with standards such as \textit{BS 5839-6:2013 Fire detection and fire alarm systems for buildings - Part 6: Code of practice for the design, installation, commissioning and maintenance of fire detection and fire alarm systems in domestic premises}.
For instance, the standard stated the different radiuses that smoke detectors and heat detectors cover; I had to thus place the detectors on the plan layouts accordingly.


\subsection*{EL6(i, b, m)}

Many courses and a couple of my placements increased my awareness of risk issues.
I learned about risk issues related to H\&S at Sunamp and in
\textit{Electrical and Lighting Services for Buildings},
\textit{Design Issues},
\textit{Inclusive and Safe Environments},
and
\textit{Laboratory Project}.
In \textit{Electrical and Lighting Services for Buildings}, for example, I learned about the importance of earthing to give \hl{overflow?} electricity an alternative path to run through which is not a person (thus electrocuting them).
I learned about environmental risk issues in
\textit{Construction Technology 1},
\textit{Introduction to the Environment},
\textit{Thermal Performance Studies},
\textit{Sustainable and Intelligent Buildings},
\textit{Climate Change, Sustainability and Adaptation},
and
\textit{Water Supply and Drainage for Buildings}.
Throughout these courses I have learned about the impact of buildings and building systems on the environment, e.g. harmful refrigerants in air conditioners, embodied carbon in building materials, and the emissions from vehicles that people use to travel to buildings.
Lastly, I learned about \hl{commercial risk (?)} in \textit{Innovation in Construction Practice}, more specifically the risk of innovation in construction projects.
\hl{Elaborate?}

I learned about the techniques of risk assessment and risk management in depth in \textit{Design Issues} and have carried out H\&S risk assessments in \textit{Facilities Management Principles}, \textit{Laboratory Project} and during my placement at Arup.
\hl{Attach LAB risk assessments?}
Generally, one first needs to identify the risks and then assess them in terms of severity and likelihood.
Afterwards, one tries to eliminate the risks (the worst first), and if that is not possible, mitigate them.


\subsection*{EL7m}

For business success, I have only gained an understanding of how innovation is a key driver in \textit{Innovation in Construction Practice} and \hl{how technology (?) plays a role in business success through the study of Fordist and post-Fordist consumerism/ manufacturing?} in \textit{Sustainable and Intelligent Buildings}.
\hl{Look up ICP and SIB notes + elaborate.}

Due to the nature of my academic programme, I have not learned as much about the key drivers for business success as I have for successful construction projects in the following courses:
\textit{Introduction to Design},
\textit{Procurement and Contracts},
\textit{Facilities Management Principles},
\hl{and more?}
However, there are two ways that this knowledge can translate to business.
Firstly, a construction project is often a means for a business to grow or improve (\hl{this is known as a secondary business case? see FM notes}).
Therefore, a successful construction project can lead to a successful business.
Secondly, the principles behind a successful construction project should also be able to be applied to businesses.
For example, after the construction sector was pointed out in the 1990s for repeatedly delivering projects that were of poor quality, overtime and over-budget, the sector has made significant efforts to improve its performance and improve client satisfaction.
Such efforts have resulted in changing the workflow so that more planning is done at the start of a project, when making changes is more flexible and less costly (\hl{get figure from ICP?}).
This includes developing more descriptive briefs, which set out the client's needs and how the construction project should be run.
\hl{I imagine this could translate in such a way to business... think!}


