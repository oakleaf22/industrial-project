
%----------------------------------------------------------------------------------------
%	SECTION 4
%----------------------------------------------------------------------------------------

\section{Hoare Lea, August 2017}

%\subsection{Electrical Engineering Summer Placement at Hoare Lea in London}

Having recently studied \textit{Electrical and Lighting Services for Buildings} and already tried out mechanical engineering at Arup, I wanted to explore the other half of M\&E.
I got the opportunity to explore electrical engineering at Hoare Lea, a building services consultancy, in their London office.
%Just preparing for my interview for the Hoare Lea placement made me take some time to reflect on skills and attributes.
%It also improved on my presentation-making skills.
Some of the tasks I got to do are:
\begin{itemize}%
	\item Produce RIBA Stage 3 layouts and schematics of electrical and fire detection and alarm installations for a new-build. These were made to comply with standards e.g. BS 5839-8:2013.
	\item Assist the Building Information Modelling (BIM) department with coordination work on Revit.
\end{itemize}

I enjoyed drawing the fire detection and alarm drawings, which was an unfamiliar exercise to me.
I learned a lot from two fire specialists from Honeywell, who guided me through the design exercise and to the relevant technical documentation.

Unfortunately due to having not-so-engaged supervisors, I do not think I learned as much about electrical engineering as I could have and would have liked.

%I practised drawing an electrical schematic by hand, on AutoCAD and on Amtech.
%Despite not doing much, I did learn a bit about busbars, lighting and resilient cables.



The London office housed many different disciplines.
I therefore took the opportunity to `roam' around and explore the different types of work that I could get into after university.
I was introduced to eight different specialities:
electrical engineering,
fire engineering,
building optimisation,
BIM,
acoustics,
sustainability,
façade access and
vertical transportation.
I was excited to learn about the breadth of career opportunities within building services.
I gained particular interest in the building optimisation speciality where one analyses the modelled and actual performance of buildings, advises clients on how to optimise the performance of their building and reduce running costs.
I would say my experience at Hoare Lea broadened rather than deepened my knowledge in AE.
Thanks to my exposure to these specialities, I have been able to discover new and exciting career paths, of which I am considering working in building optimisation (for more about this, go to Section~\ref{sec:future}).