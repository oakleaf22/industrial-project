
%----------------------------------------------------------------------------------------
%	SECTION 1
%----------------------------------------------------------------------------------------


\section{Hultin \& Lundquist Arkitekter, October - December 2013}

%\subsection{Architecture Placement at Hultin \& Lundquist Arkitekter in Malm\"o, Sweden}

During the first half of my gap year, which I spent in Sweden, I was keen to gain work experience in architecture. I applied to all the architectural companies I could find in the Yellow Pages, with help from \textit{Ung Vision} (Young Vision).
\textit{Ung Vision} is a local initiative funded by Svedala council
%\footnote{\href{https://www.svedala.se/}{svedala.se}} 
with support from \textit{Arbetsf\"ormedlingen} (the Swedish Public Employment Service)
%\footnote{\href{https://www.arbetsformedlingen.se/Globalmeny/Other-languages/Languages/English-engelska.html}{arbetsformedlingen.se in English}} 
and \textit{F\"orsäkringskassan} (the Swedish Social Insurance Agency)
%\footnote{\href{https://www.forsakringskassan.se/privatpers/!ut/p/z1/04_Sj9CPykssy0xPLMnMz0vMAfIjo8ziTTxcnA3dnQ28LdyNTQ0cAwMMjU38jby8gg30w_Wj9KOASgxwAEcD_YLsbEUAFUIRCA!!/dz/d5/L0lDUmlTUSEhL3dHa0FKRnNBLzROV3FpQSEhL2Vu/?keepNavState=true}{forsakringskassan.se in English}} 
that aims to help unemployed people between the ages of 16 and 24 find work \citep{ungvision:online}.
I eventually managed to get myself a ten-week paid placement at Hultin \& Lundquist Arkitekter, a small architectural practice in Malm\"o which at the time consisted of two architects.

Working with buildings was very exciting for me; I would walk proudly with my clipboard, floor plan drawings and ruler to the 1960s apartment building the architects were going to refurbish.
I was responsible for correcting the measurements of the building's outdated architectural drawings.
I learned some of the rules used in architectural drawings and also gained some 2D modelling skills in  AutoCAD.
One of the things that still stands out to me today was how the building layout responded to the needs of the occupants in the 1960s.
Many flats had two entrances, which puzzled me until I was told that the larger door was the main entrance and the smaller a discreet `back' entrance for the servants.
Such nuggets of history that explained the purpose behind design was the beginning of my learning how design should be purposeful.
It also influenced me to apply for an architecture-related programme at university.