%----------------------------------------------------------------------------------------
%	SECTION 5
%----------------------------------------------------------------------------------------

\section{Engineering Practice (P)}


\subsection*{P1(i, -) \hl{fix table so entire enumerate list not affected}}

\begin{wraptable}{r}{0.2\textwidth}
    \begin{tabular}{|ll|}
        \hline
        \multicolumn{2}{|c|}{\cellcolor[HTML]{F8A102}\textbf{P1(i, -)}} \\ \hline
        \BST & \IE \\
        \DPA & \Acoustics \\
        \EnvBeh & \EPA \\
        \DPB & \CAS \\
        \ELS & \EnBldgs \\
        \TPS & \DI \\
        \FMP & \PRJ \\
        \DST & \LAB \\
        \SIB & \CCSA \\
        \WSD &  \\
        Arup & Hoare Lea \\
        Sunamp & Sweco \\ \hline
    \end{tabular}
\end{wraptable}

Throughout the programme, I have gained knowledge and understanding of some contexts in which my AE knowledge can be applied.
The following list, in which I describe such contexts, is not exhaustive.

\begin{enumerate}
	\item %The collaborative design projects in years \hl{1/2} and 4,
	%\textit{Building Services Technology},
	%\textit{Introduction to the Environment},
	%\textit{Design Project A},
	%\textit{Design Project B},
	%\textit{Electrical and Lighting Services for Buildings},
	%\textit{Critical Architectural Studies},
	%\textit{Design Project},  
	Several courses throughout the years and my placements at Arup and Hoare Lea
	provided me with the practical experience of applying my AE knowledge in the context of a team designing a building and/ or its services, in which I was responsible for the building services, the building's internal environment and the occupants' comfort levels.
	My group's first prize for sustainability (see Figure \ref{fig_award}) demonstrates my understanding of an Architectural Engineer's role in passive design (e.g. natural lighting).
	
	\item Throughout several courses (especially \FMPTitle),
	%\textit{Acoustics and Architectural Design},
	%\textit{Environment and Behaviour},
	%\textit{Energy Principles and Applications},
	%\textit{Energy and Buildings},
	%\textit{Thermal Performance Studies},
	%\textit{Design Issues},
	%\textit{Facilities Management Principles},
	%\textit{Laboratory Project},
	%\textit{Sustainable and Intelligent Buildings},
	%\textit{Climate Change, Sustainability and Adaptation},
	%\textit{Water Supply and Drainage for Buildings},
	my work in the environmental certification of existing buildings at Sweco,
	and
	my encounter with a specialist in Performance at Hoare Lea,
	%Design Issues,
	I have also learned that my AE knowledge is applicable in the operation and management of buildings and building services (a.k.a. facilities management) to, for example, improve the occupants' comfort levels, optimise performance or increase resilience to future impacts of climate change.
	
	\item Through the work I did at Sunamp as well as my encounter with a technical author at Hoare Lea, I understand that engineering knowledge is important in the composition of technical documents (e.g. specifications, operation and maintenance manuals) and even marketing material for technical products, which may need to explain engineering processes graphically and in layman's terms.
	See how I did this in the UniQ PISs in Appendix \ref{App:PISs}.
	
	\item My dissertation in particular taught me that engineering knowledge is necessary for the development of industry standards and codes of practice, such as the series of Publicly Available Specifications (PAS) 1192 which standardise the requirements for achieving BIM Level 2.
	
	\item Throughout my dissertation and \textit{Climate Change, Sustainability and Adaptation} (\CCSA), I have also come to understand that my engineering knowledge can be used for research, the development of new technologies or processes, and to influence policies.
	One of the assignments for \CCSA \space was to write a briefing for politicians;
	mine urged them to develop policies to make the existing UK housing stock resilient to future climate impacts.
	%\hl{Repeated DST and CCSA. Could even include Sunamp. Maybe this example is one too many...}
\end{enumerate}




\subsection*{P2(i, b, m) \hl{fix table so entire enumerate list not affected}}

\begin{wraptable}{r}{0.2\textwidth}
    \begin{tabular}{|ll|}
        \hline
        \multicolumn{2}{|c|}{\cellcolor[HTML]{F8A102}\textbf{P2(i, b, m)}} \\ \hline
        \ConTechOne & \BST \\
        \DPA & \ConTechTwo \\
        \Acoustics & \EnvBeh \\
        \Stats & \DPB \\
        \CAS & \ELS \\
        \DSA & \EnBldgs \\
        \TPS & \PRJ \\
        \DST & \LAB \\
        \SIB & \ICP \\
        H\&L & Arup \\
        Hoare Lea & Sunamp \\ \hline
    \end{tabular}
\end{wraptable}

I have gained knowledge of the characteristics of, an understanding of and an ability to use a range of computer-based tools, engineering processes, and building services products:
\begin{itemize}
    %\item P2i 
     
    % In \textit{Environment and Behaviour}, \textit{Laboratory Project} and \textit{Dissertation}, I developed an ability to use the statistics software tool SPSS (Statistical Package for the Social Science).
    % The course \textit{Statistics for Science} provided me with a foundation to understand the data in the inputs and outputs of this software.
    
    \item \textbf{P2i and P2b}
    
    In \textit{Design Software Applications} and \textit{Laboratory Project}, I gained knowledge of the characteristics of steady-state and dynamic building modelling and energy analysis software programmes, notably iSBEM, SAP, IES, and CFD [\textbf{P2b}].
    As these courses only provided an introduction to these software programmes, I have not mastered my ability to use them [\textbf{P2i}].
    \hl{Look up} \DSATitle \space \hl{notes if you want to demonstrate knowledge.}
    This was demonstrated during my attempt to model my \textit{Design Project} building in IES; my building model overheated due to variation and temperature profiles that I had incorrectly set up, amongst other things.
    
    %During the \textit{Design Project} and my placements at Hultin \& Lundquist Arkitekter and Sunamp, I gained an understanding of and an ability to use Autocad.
    %Likewise, I learned to use Revit during the \textit{Design Project} and my placements at Arup and Hoare Lea.
    %Although my ability to use these Autodesk products (i.e. Autocad and Revit) is limited, I have learned quite a bit about their characteristics.
    %In \textit{Dissertation} and \textit{Innovation in Construction Practice}, I learned about the prominent use of these products in the UK construction industry and how their proprietary characteristics can lead to problems of interoperability.
    
    Regarding engineering processes, 
    %I have learned extensively about the BIM process in Dissertation and Innovation in Construction Practice.
    I have learned the characteristics of and practised the optimisation process of an engineering design in \textit{Laboratory Project} (see \textit{EA3(i, b, m)} in Section \ref{EA3} for more detail) [\textbf{P2i and P2b}].
    
    \item \textbf{P2m}
    
    Throughout many of my construction-based courses and my placements at Sunamp, Hoare Lea and Arup,
    I have accumulated an extensive knowledge of a wide range of building services products (e.g. heat pumps, heat batteries, PV panels) and building materials (e.g. PCMs, Ziegel blocks, steel).
    My understanding of these products and materials, however, varies.
    \hl{Elaborate with heat pump drawing and compare with Ziegel blocks, for example.}
    %throughout many of my construction-based courses, i.e. 
    %\textit{Construction Technology 1},
    %\textit{Building Services Technology},
    %\textit{Design Project A},
    %\textit{Construction Technology 2},
    %\textit{Acoustics and Architectural Design},
    %\textit{Design Project B},
    %\textit{Critical Architectural Studies},
    %\textit{Electrical and Lighting Services for Buildings},
    %\textit{Design Software Applications},
    %\textit{Energy and Buildings},
    %\textit{Thermal Performance Studies},
    %\textit{Design Project},
    %\textit{Laboratory Project},
    %\textit{Sustainable and Intelligent Buildings},
    %\textit{Innovation in Construction Practice},
    %and \textit{Water Supply and Drainage for Buildings},
    %and my placements at Sunamp, Hoare Lea and Arup
    
    \hl{Move following explanation to chapter intro?}
    N.B. P2m is a LO for Masters students.
    However, I do not think it is reasonable to assume that somebody can gain an extensive knowledge and understanding of a wide range of \hl{engineering materials and components/ anything} exclusively at Masters level \hl{create two graphs to illustrate learning curve?}.
    I believe this is a cumulative process, which is why, in my \hl{Skills Matrix (see ...}), I have marked courses since Year 1 as having contributed to my attainment of this LO.
\end{itemize}
%\hl{Write about increase in knowledge throughout the years. P2m for example couldn't all of a sudden have been achieved in Y5.}

%Building materials: insulation, steel, wood, Ziegel blocks, glass, PCMs etc.

%Building services products: ACs, HPs, PV, AHUs, Sunamp, windcatchers




\subsection*{P3(i, -)}

\begin{wraptable}{r}{0.2\textwidth}
    \begin{tabular}{|ll|}
        \hline
        \multicolumn{2}{|c|}{\cellcolor[HTML]{F8A102}\textbf{P3(i, -)}} \\ \hline
        \ConTechTwo & \Acoustics \\
        \HYD & \EPA \\
        \CAS & \TPS \\
        \LAB &  \\ \hline
    \end{tabular}
\end{wraptable}

%\begin{itemize}
%    \item \textbf{P3i}
    
    I have already had laboratory practice in physics, chemistry and biology in high school.
    In university, my knowledge and understanding of laboratory practice began to further develop in Year 2 when I conducted experiments in \textit{Construction Technology 2}, \textit{Acoustics and Architectural Design}, \textit{Hydraulics and Hydrology A} and \textit{Energy Principles and Applications} [\textbf{P3i}].
    The feedback for these lab reports (\hl{respectively ..., 93\% and ... cannot find other marks}) demonstrate my knowledge and understanding of laboratory practice.
    
%    \item \textbf{P3}
    
    In Years 3 and 4, I gained the ability to apply practical and laboratory skills that were more relevant to AE through my increased exposure to and application of laboratory practice in thermofluids and acoustics [\textbf{P3}].
    I re-visited the Services lab another two times for thermofluids-related experiments in \textit{Thermal Performance Studies} and \textit{Laboratory Project} and re-used my practical skills of recording, measuring and analysing sounds in \textit{Critical Architectural Studies} and \textit{Laboratory Project}.
%\end{itemize}

My final mark of 73\% in the \LABTitle \space demonstrates my full achievement of LOs P3i and P3.
%\hl{What actions and results can I show as evidence of my practical skills?}





\subsection*{P4(i, -)}

\begin{wraptable}{r}{0.2\textwidth}
    \begin{tabular}{|l|}
        \hline
        \rowcolor[HTML]{F8A102} 
        \multicolumn{1}{|c|}{\cellcolor[HTML]{F8A102}\textbf{P4(i, -)}} \\ \hline
        \PRJ \\
        Sunamp \\ \hline
    \end{tabular}
\end{wraptable}

I have developed the ability to use and apply information from technical literature during my \PRJTitle \space and my Sunamp placement [\textbf{P4i}].
For the \PRJTitle, I sought, used and applied information from product specifications in order to size my building's water storage tank \citep{Decca}, calorifiers and buffer vessels \citep{RycroftLtd} etc.
At Sunamp, I used their technical manuals in order to produce the UniQ PISs (Appendix \ref{App:PISs}).
%The technical literature I have been exposed to includes Sunamp's manuals for their UniQ product range and the product information sheets I drew up for them.
These sheets included information such as temperature input and output ranges, schematic diagrams, and the dimensions of the heat batteries.
%\hl{Include evidence, e.g. attach PISs}

%I also came across and used technical literature during my \textit{Design Project}.
%I used product specifications to size the college's water storage tank \citep{Decca}, calorifiers and buffer vessels \citep{RycroftLtd} etc.

Throughout my \textit{Design Project} and Sunamp experiences, I gained an understanding of the use of technical literature [\textbf{P4}].
%During the Design Project, I used technical literature to appropriately size some of the building services, both to accommodate the needs of the building's occupants and to fit inside of the plant room.
For example, at Sunamp, I was part of the decision-making process of the information that should be included in the UniQ PISs, which would be one of the first documents a customer would see once they have taken interest in Sunamp's products.
It was decided that information such as dimensions and weight were important to include for a couple reasons.
Firstly, to highlight the compact size of Sunamp's batteries (which is one of their unique selling points).
Secondly, to give the customers an indication of the transportation and placement requirements of a battery (e.g. depending on the type and size, a single UniQ battery will weigh between 55 kg and 211 kg 
%be it may need to be carried by two people and 
and will thus need to be placed on a load-bearing floor).
%\hl{Perhaps do not be afraid to write more sentences and give details such as specific weights and dimensions, and talk about stacking batteries and load bearing floors vs shelves etc.}


\subsection*{P5}

Considering that in \hl{AHEP} defines ``knowledge" as information that can be recalled, I do not have much knowledge of legal and contractual issues relevant to AE.
We have, however, covered \hl{contracts...} in \textit{Procurement and Contracts} and professional liability and insurance in \textit{Design Issues}.
It is perhaps due to a lack of coursework or examination and/ or a lack of spaced repetition of these topics throughout my degree programme that have I not gained a firm knowledge in relevant legal and contractual issues.


\subsection*{P6(i, -)}

I have used appropriate codes of practice and industry standards during \textit{Design Project}, \textit{Inclusive and Safe Environments} and my placements at Hoare Lea and Sweco.
For the \textit{Design Project}, for example, I wanted to use water source heat pumps (WSHPs) to generate heat for the college since it was located right next to a body of water \hl{(see Figure ...).}
I had to figure out how WSHPs could fit into the building's heating strategy.
To do this, I read about WHSPs in CIBSE CP2, the UK's Code of Practice for surface WSHPs.
I decided, because the college was a large non-domestic facility, to use a multi-valent heating system, where the WSHPs would provide the base heating load and some micro combined heat and power (CHP) units would satisfy the peak heating demands \citep[pp.~12,~38]{CP22016}.
\hl{I have no feedback to demonstrate this specifically was good and my grade for DP wasn't good because of missing elements...}


\subsection*{P7}


\subsection*{P8}

I think part of \hl{my failure?} of the Design Project was my inability to work with technical uncertainty.
I have a tendency to be detail-oriented and thus get stuck in details etc., to not take risks and to be indecisive
But I did (painstakingly) work through some uncertainties, thus achieving a passing grade (?).


\subsection*{P9m}


\subsection*{P10m}


\subsection*{P11(i, b, m)}


