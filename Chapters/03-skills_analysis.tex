\chapter{Skills Analysis} % Main chapter title

\label{Chapter3} % Change X to a consecutive number; for referencing this chapter elsewhere, use \ref{ChapterX}

\lhead{Chapter 3. \emph{Skills Analysis}} % Change X to a consecutive number; this is for the header on each page - perhaps a shortened title


% Please add the following required packages to your document preamble:
% \usepackage{booktabs}
% \usepackage{multirow}
\begin{table}[]
\caption{The courses I took towards my MEng degree in Architectural Engineering at Heriot-Watt University (include course codes? transpose table!)}
\label{courses}
	\begin{tabular}{@{}lp{8cm}p{7.5cm}@{}}
		\toprule
		& Semester 1 & Semester 2 \\ \midrule
		\multirow{4}{*}{\rot{Year 1}} & \textbullet \hspace{0.5ex}Construction Technology 1 & \textbullet \hspace{0.5ex}Building Services Technology \\
		& \textbullet \hspace{0.5ex}History of the Built Environment & \textbullet \hspace{0.5ex}Mechanics B \\
		& \textbullet \hspace{0.5ex}Introduction to Design & \textbullet \hspace{0.5ex}Introduction to the Environment \\
		& \textbullet \hspace{0.5ex}Mathematics for Engineers and Scientists 1 & \textbullet \hspace{0.5ex}Mathematics for Engineers and Scientists 2 \\ \midrule
		\multirow{4}{*}{\rot{Year 2}} & \textbullet \hspace{0.5ex}Design Project A & \textbullet \hspace{0.5ex}Environment and Behaviour \\
		& \textbullet \hspace{0.5ex}Construction Technology 2 & \textbullet \hspace{0.5ex}Statistics for Science \\
		& \textbullet \hspace{0.5ex}Acoustics and Architectural Design & \textbullet \hspace{0.5ex}Energy Principles and Applications \\
		& \textbullet \hspace{0.5ex}\textit{Hydraulics \& Hydrology A} & \textbullet \hspace{0.5ex}\textit{Design Project B} \\ \midrule
		\multirow{4}{*}{\rot{Year 3}} & \textbullet \hspace{0.5ex}Critical Architectural Studies & \textbullet \hspace{0.5ex}Energy and Buildings \\ 
		& \textbullet \hspace{0.5ex}Procurement and Contracts & \textbullet \hspace{0.5ex}Thermal Performance Studies \\
		& \textbullet \hspace{0.5ex}Design Software Applications & \textbullet \hspace{0.5ex}Design Issues \\
		& \textbullet \hspace{0.5ex}Electrical and Lighting Services for Buildings & \textbullet \hspace{0.5ex}Facilities Management Principles \\ \midrule
		\multirow{4}{*}{\rot{Year 4}} & \textbullet \hspace{0.5ex}Design Project (S1) & \textbullet \hspace{0.5ex}Design Project (S2) \\
		& \textbullet \hspace{0.5ex}Dissertation (S1) & \textbullet \hspace{0.5ex}Dissertation (S2) \\
		& \textbullet \hspace{0.5ex}Laboratory Project & \textbullet \hspace{0.5ex}Sustainable and Intelligent Buildings \\
		& \textbullet \hspace{0.5ex}Inclusive and Safe Environments & \textbullet \hspace{0.5ex}Innovation in Construction Practice \\ \midrule
		\multirow{4}{*}{\rot{Year 5}} & \textbullet \hspace{0.5ex}Industrial Project & \textbullet \hspace{0.5ex}Architectural Acoustics \\
		& \textbullet \hspace{0.5ex}\textit{Water Supply \& Drainage for Buildings} & \textbullet \hspace{0.5ex}Design of Low Carbon Buildings \\
		& \textbullet \hspace{0.5ex}\textit{Climate Change, Sustainability \& Adaptation} & \textbullet \hspace{0.5ex}Thermofluids \\
		&  & \textbullet \hspace{0.5ex}\textit{OPTIONAL} \\ \bottomrule
	\end{tabular}
\end{table}


%----------------------------------------------------------------------------------------
%	SECTION 1
%----------------------------------------------------------------------------------------

\section{Science and Mathematics (SM)}

\subsection*{SM1(i, b, m)}

Multiple courses (mainly from years 2 and 3) have contributed to my knowledge and understanding of scientific principles and methodology necessary to underpin my education in Architectural Engineering.
A main contributor was \textit{Thermal Performance Studies}.
On this course I learned the scientific principles of psychrometrics, i.e. the properties of air-vapour mixtures (especially/ like air), and how to use this knowledge in the design of air conditioners, heat exchangers and cooling towers.
Learning about the environmental effects of refrigerants has contributed to my understanding of the evolution of air conditioners, from using carbon and fluoride based refrigerants to water and eventually phase change materials (PCMs) like Sunamp uses.

Learning the scientific principles of related disciplines have helped me enrich my appreciation of the scientific and engineering context of my discipline.
\textit{Statistics for Science} has, for example, helped me understand the proper methodology behind sampling populations and to identify when bias in data.
\textit{Hydraulics and Hydrology A}, a civil engineering course, helped me appreciate the hydrological cycle (relevant to rainwater/ urban drainage) and water flow in pipes (relevant to drainage, water supply and heating systems).


\subsection*{SM2(i, b, m)}

\textit{Mathematics for Engineers and Scientists 1 and 2} and \textit{Statistics for Science} all further developed my knowledge and understanding of mathematical and statistical methods.
I have been able to apply this knowledge in the analysis and solution of engineering problems.
For example, I used it to work out the yield of an array of photovoltaic panels (PV) in \textit{Energy and Buildings} as well as the yield of a set of turbines in a tidal energy generation system as part of my group project in \textit{Design Project}.


\subsection*{SM3(b, m)}

Other engineering disciplines that I have gained knowledge and understanding from are predominantly civil and structural.
These were acquired through the following courses: \textit{Mechanics B}, \textit{Construction Technology 1 and 2},  and \textit{Hydraulics and Hydrology A}.
\hl{I cannot remember how I have applied or integrated this knowledge though...
pipes?
materials choice?}


\subsection*{SM4m}

A range of courses increased my awareness of developing technologies relevant to Architectural Engineering and building services.
In \textit{Building Services Technology}, \textit{Energy and Buildings}, \textit{Dissertation} and \textit{Innovation in Construction Practice}, I respectively learned about technologies such as modern windcatchers, smart meters, Building Information Modelling (BIM) and virtual and augmented reality.
I was also exposed to a developing technology during my placement at Sunamp: their heat batteries.
\hl{refer to work done at Sunamp?}


\subsection*{SM5m}

\hl{Is this skill about software???}
\textit{Design Software Applications} and \textit{Laboratory Project} enabled me to gain a comprehensive knowledge and understanding of software used by architectural engineers.
I learned how steady-state and dynamic software applications function, was made aware of the mathematical \hl{models/ formulas} they are based on and learned about their limitations.
\hl{Should I give examples of limitations to demonstrate my knowledge/ understanding? Not necessarily part of marking criteria.}
I was introduced to and got to practise using the following applications:
\begin{itemize}
    \item Steady-state: iSBEM (an interface for Simplified Building Energy Model) and SAP (Standard Assessment Procedure)
    \item Dynamic: IES-VE (Integrated Environmental Solutions - Virtual Environment)
    \item Computational fluid dynamics (CFD): PHOENICS
\end{itemize}

% Please add the following required packages to your document preamble:
% \usepackage{booktabs}
\begin{table}[htbp]
\begin{tabular}{@{}lp{8cm}@{}}
\toprule
Type of software & Applications \\ \midrule
Steady-state & iSBEM (an interface for Simplified Building Energy Model) \\
 & SAP (Standard Assessment Procedure) \\
Dynamic & IES-VE (Integrated Environmental Solutions - Virtual Environment) \\
Computational fluid dynamics (CFD) & PHOENICS \\ \bottomrule
\end{tabular}
\end{table}

\hl{List or table? If table, what is best positioning: htbp?}

\hl{Although it is outside the scope of this learning outcome, I would like to add that these courses did not allow me to properly develop my skills in using the software...}


\subsection*{SM6m}

Use of solar gain in pool house (first year collaborative project)?

Sustainability hierarchy/ concept effectively used in CAS project.

Contact factor in LAB CFD?

Passive design? Design Project?


%----------------------------------------------------------------------------------------
%	SECTION 2
%----------------------------------------------------------------------------------------

\section{Engineering Analysis (EA)}


%----------------------------------------------------------------------------------------
%	SECTION 3
%----------------------------------------------------------------------------------------

\section{Design (D)}


%----------------------------------------------------------------------------------------
%	SECTION 4
%----------------------------------------------------------------------------------------

\section{Economic, Legal, Social, Ethical and Environmental Context (EL)}


%----------------------------------------------------------------------------------------
%	SECTION 5
%----------------------------------------------------------------------------------------

\section{Engineering Practice (P)}


%----------------------------------------------------------------------------------------
%	SECTION 6
%----------------------------------------------------------------------------------------

\section{Additional General Skills (G)}