\chapter{Skills Analysis} % Main chapter title

\label{Chapter3} % Change X to a consecutive number; for referencing this chapter elsewhere, use \ref{ChapterX}

\lhead{Chapter 3. \emph{Skills Analysis}} % Change X to a consecutive number; this is for the header on each page - perhaps a shortened title


% Please add the following required packages to your document preamble:
% \usepackage{booktabs}
% \usepackage{multirow}
\begin{table}[]
\caption{The courses I took towards my MEng degree in Architectural Engineering at Heriot-Watt University (include course codes? transpose table!)}
\label{courses}
	\begin{tabular}{@{}lp{8cm}p{7.5cm}@{}}
		\toprule
		& Semester 1 & Semester 2 \\ \midrule
		\multirow{4}{*}{\rot{Year 1}} & \textbullet \hspace{0.5ex}Construction Technology 1 & \textbullet \hspace{0.5ex}Building Services Technology \\
		& \textbullet \hspace{0.5ex}History of the Built Environment & \textbullet \hspace{0.5ex}Mechanics B \\
		& \textbullet \hspace{0.5ex}Introduction to Design & \textbullet \hspace{0.5ex}Introduction to the Environment \\
		& \textbullet \hspace{0.5ex}Mathematics for Engineers and Scientists 1 & \textbullet \hspace{0.5ex}Mathematics for Engineers and Scientists 2 \\ \midrule
		\multirow{4}{*}{\rot{Year 2}} & \textbullet \hspace{0.5ex}Design Project A & \textbullet \hspace{0.5ex}Environment and Behaviour \\
		& \textbullet \hspace{0.5ex}Construction Technology 2 & \textbullet \hspace{0.5ex}Statistics for Science \\
		& \textbullet \hspace{0.5ex}Acoustics and Architectural Design & \textbullet \hspace{0.5ex}Energy Principles and Applications \\
		& \textbullet \hspace{0.5ex}\textit{Hydraulics \& Hydrology A} & \textbullet \hspace{0.5ex}\textit{Design Project B} \\ \midrule
		\multirow{4}{*}{\rot{Year 3}} & \textbullet \hspace{0.5ex}Critical Architectural Studies & \textbullet \hspace{0.5ex}Energy and Buildings \\ 
		& \textbullet \hspace{0.5ex}Procurement and Contracts & \textbullet \hspace{0.5ex}Thermal Performance Studies \\
		& \textbullet \hspace{0.5ex}Design Software Applications & \textbullet \hspace{0.5ex}Design Issues \\
		& \textbullet \hspace{0.5ex}Electrical and Lighting Services for Buildings & \textbullet \hspace{0.5ex}Facilities Management Principles \\ \midrule
		\multirow{4}{*}{\rot{Year 4}} & \textbullet \hspace{0.5ex}Design Project (S1) & \textbullet \hspace{0.5ex}Design Project (S2) \\
		& \textbullet \hspace{0.5ex}Dissertation (S1) & \textbullet \hspace{0.5ex}Dissertation (S2) \\
		& \textbullet \hspace{0.5ex}Laboratory Project & \textbullet \hspace{0.5ex}Sustainable and Intelligent Buildings \\
		& \textbullet \hspace{0.5ex}Inclusive and Safe Environments & \textbullet \hspace{0.5ex}Innovation in Construction Practice \\ \midrule
		\multirow{4}{*}{\rot{Year 5}} & \textbullet \hspace{0.5ex}Industrial Project & \textbullet \hspace{0.5ex}Architectural Acoustics \\
		& \textbullet \hspace{0.5ex}\textit{Water Supply \& Drainage for Buildings} & \textbullet \hspace{0.5ex}Design of Low Carbon Buildings \\
		& \textbullet \hspace{0.5ex}\textit{Climate Change, Sustainability \& Adaptation} & \textbullet \hspace{0.5ex}Thermofluids \\
		&  & \textbullet \hspace{0.5ex}\textit{OPTIONAL} \\ \bottomrule
	\end{tabular}
\end{table}


%----------------------------------------------------------------------------------------
%	SECTION 1
%----------------------------------------------------------------------------------------

\section{Science and Mathematics (SM)}

\subsection*{SM1(i, b, m)}

Multiple courses (mainly from years 2 and 3) have contributed to my knowledge and understanding of scientific principles and methodology necessary to underpin my education in Architectural Engineering.
A main contributor was \textit{Thermal Performance Studies}.
On this course I learned the scientific principles of psychrometrics, i.e. the properties of air-vapour mixtures (especially/ like air), and how to use this knowledge in the design of air conditioners, heat exchangers and cooling towers.
Learning about the environmental effects of refrigerants has contributed to my understanding of the evolution of air conditioners, from using carbon and fluoride based refrigerants to water and eventually phase change materials (PCMs) like Sunamp uses.

Learning the scientific principles of related disciplines have helped me enrich my appreciation of the scientific and engineering context of my discipline.
\textit{Statistics for Science} has, for example, helped me understand the proper methodology behind sampling populations and to identify when bias in data.
\textit{Hydraulics and Hydrology A}, a civil engineering course, helped me appreciate the hydrological cycle (relevant to rainwater/ urban drainage) and water flow in pipes (relevant to drainage, water supply and heating systems).


\subsection*{SM2(i, b, m)}

\textit{Mathematics for Engineers and Scientists 1 and 2} and \textit{Statistics for Science} all further developed my knowledge and understanding of mathematical and statistical methods.
I have been able to apply this knowledge in the analysis and solution of engineering problems.
For example, I used it to work out the yield of an array of photovoltaic panels (PV) in \textit{Energy and Buildings} as well as the yield of a set of turbines in a tidal energy generation system as part of my group project in \textit{Design Project}.


\subsection*{SM3(b, m)}

Other engineering disciplines that I have gained knowledge and understanding from are predominantly civil and structural.
These were acquired through the following courses: \textit{Mechanics B}, \textit{Construction Technology 1 and 2},  and \textit{Hydraulics and Hydrology A}.
%\hl{I cannot remember how I have applied or integrated this knowledge though...
%pipes?
%materials choice?}
I have been able to apply and integrate this knowledge and understanding to support study of my own engineering discipline, Architectural Engineering.
For example, during a \hl{CFD (1st time I mention CFD?)} modelling exercise in Laboratory Project, I was able to identify laminar and turbulent flows, something I had learned about it in \textit{Hydraulics and Hydrology A}.
This enabled me to refine and improve my CFD model.


\subsection*{SM4m}

A range of courses increased my awareness of developing technologies relevant to Architectural Engineering and building services.
In \textit{Building Services Technology}, \textit{Energy and Buildings}, \textit{Dissertation} and \textit{Innovation in Construction Practice}, I respectively learned about technologies such as modern windcatchers, smart meters, Building Information Modelling (BIM) and virtual and augmented reality.
I was also exposed to a developing technology during my placement at Sunamp: their heat batteries.
\hl{refer to work done at Sunamp?}


\subsection*{SM5m}

\hl{Is this skill about software???}
\textit{Design Software Applications} and \textit{Laboratory Project} enabled me to gain a comprehensive knowledge and understanding of software used by architectural engineers.
I learned how steady-state and dynamic software applications function, was made aware of the mathematical \hl{models/ formulas} they are based on and learned about their limitations.
\hl{Should I give examples of limitations to demonstrate my knowledge/ understanding? Not necessarily part of marking criteria.}
I was introduced to and got to practise using the following applications:
\begin{itemize}
    \item Steady-state: iSBEM (an interface for Simplified Building Energy Model), SAP (Standard Assessment Procedure) and PHOENICS (a CFD software application)
    \item Dynamic: IES-VE (Integrated Environmental Solutions - Virtual Environment)
    \item Computational fluid dynamics (CFD): PHOENICS?
\end{itemize}

% Please add the following required packages to your document preamble:
% \usepackage{booktabs}
\begin{table}[htbp]
\begin{tabular}{@{}lp{8cm}@{}}
\toprule
Type of software & Applications \\ \midrule
Steady-state & iSBEM (an interface for Simplified Building Energy Model) \\
 & SAP (Standard Assessment Procedure) \\
 & PHOENICS (a CFD software application) \\
Dynamic & IES-VE (Integrated Environmental Solutions - Virtual Environment) \\
Computational fluid dynamics (CFD)? & PHOENICS? \\ \bottomrule
\end{tabular}
\end{table}

\hl{List or table?}

\hl{Although it is outside the scope of this learning outcome, I would like to add that these courses did not allow me to properly develop my skills in using the software...}


\subsection*{SM6m}

\hl{How does this fall under Science and Maths heading?}

I have been able to critically evaluate and apply a range of concepts, including some outside engineering, in my engineering projects.
\hl{EXAMPLE 1:}
I learned about the sustainable design hierarchy (see Figure \ref{fig_hierarchy}) during my placement at Arup.
I then applied this approach to the design of a library for a group project in \textit{Critical Architectural Studies}, helping us earn the first prize in Sustainable Design (see Figure \ref{fig_award}).
\hl{EXAMPLE 2:}
Upon conducting a post-occupancy evaluation (POE) for \textit{Environment and Behaviour}, I came across a social sciences journal paper about survey research.
It analysed typical response styles and suggested how to design questionnaires to avoid biased or inaccurate data collection.
Using these \hl{criteria/ suggestions}, I was able to critically evaluate the success of the questionnaire used for the POE in \textit{Environment and Behaviour} and design an improved survey questionnaire for the acoustic \textit{Laboratory Project}.

\begin{figure}[htbp]
	\centering
	\includegraphics[width=10cm]{figures/Hierarchy.jpg}
	\rule{\textwidth}{0.5pt} % use line???
	\caption{The sustainable design hierarchy \citep{Dougherty:online}.}
	\label{fig_hierarchy}
\end{figure}


\begin{figure}[htbp]
	\centering
	\includegraphics[width=\textwidth]{figures/SustainabilityAward.png}
	\rule{\textwidth}{0.5pt} % use line???
	\caption{First prize for sustainability awarded to my group for our library design in \textit{Critical Architectural Studies} (course code: D19CX).}
	\label{fig_award}
\end{figure}

\hl{Other courses?}
\begin{itemize}
	\item Use of solar gain in pool house (\textit{first year collaborative project})?
	\item Contact factor in \textit{LAB CFD}?
	\item Learned in \textit{CAS}, applied again in \textit{y4 collab}: make design concept red thread...
\end{itemize}


%----------------------------------------------------------------------------------------
%	SECTION 2
%----------------------------------------------------------------------------------------

\section{Engineering Analysis (EA)}

\subsection*{EA1(i, b, m)}

Conducting laboratory experiments and modelling exercises in \textit{Energy Principles and Applications}, \textit{Design Software Applications}, \textit{Thermal Performance Studies} and \textit{Laboratory Project} further developed my skills in monitoring, interpreting and applying the results of analysis to bring about continuous improvement (EA1i).
\textit{Laboratory Project} went a step further and taught me the principles and process to engineering a better solution (EA1b and EA1m).
These principles were applied in an engineering project which aimed to optimise the design of a cooling coil.
\hl{Analysis of key engineering processes???}


\subsection*{EA2(i, -)}

A range of courses have developed my ability to understand and explain the performance of systems and components through the use of quantitative/ analytical methods and/ or modelling techniques:
\textit{Construction Technology 2},
\textit{Acoustics and Architectural Design},
\textit{Hydraulics and Hydrology A},
\textit{Energy Principles and Applications},
\textit{Thermal Performance Studies},
\textit{Laboratory Project},
and \textit{Design Project}.
For example, in \textit{Hydraulics and Hydrology A}, I learned how to use quantitative methods to distinguish laminar flows from turbulent flows.
And when we were studying thermofluids in \textit{Laboratory Project}, I used both CFD modelling and analytical methods to understand and describe the performance of a cooling coil in an air conditioning unit.

I have also applied quantitative/ analytical methods and/ or modelling techniques to classify and describe the performance of systems during my work placements at Arup and Sweco.
In particular, at Sweco I used a combination of the aforementioned methods and techniques to describe the performance of some buildings in terms of solar heat gain, energy consumption, thermal comfort in winter and summer, and daylighting.
The results of these analyses were then classified according to the criteria of \textit{Miljöbyggnad}, i.e. the Swedish environmental certification system for buildings: Gold, Silver, Bronze or Disapproved.


\subsection*{EA3(i, b, m)}

I have developed my ability to use the results of analysis and apply quantitative and computational methods to solve engineering problems and subsequently recommend or implement appropriate action in the following courses:
\textit{Laboratory Project},
\hl{...}
For example, a combination of calculations, CFD modelling and a physical laboratory experiment were used to optimise the design of a cooling coil in \textit{Laboratory Project}.
This was an iterative process of solving problems and implementing appropriate actions to come up with a better solution.


\subsection*{EA4(i, b, m)}

\hl{Don't understand}


\subsection*{EA5m}

In addition to my work placements at Hoare Lea and Sunamp, courses that I have developed my ability to use fundamental knowledge to investigate new and emerging technologies are
\textit{Energy and Buildings},
\textit{Innovation in Construction Practice},
\textit{Design Project},
and \textit{Dissertation}.
My most extensive investigation must be of Sunamp's heat batteries, having written the information sheets for their UniQ product range.


\subsection*{EA6m}

\hl{The only example I can think of is when I tried to calculate the yield of tidal power.
This was an unfamiliar problem and I could only find information online on how to calculate yields from hydroelectric turbines (?).
Thus, I had to use a combination of the information I found, assumptions and my own fundamental mathematical skillset.
Still, I never showed my workings to anyone, so I can't be sure if they were correct.

Fan's physics/ mathematical exercises at start of TPS?}

%----------------------------------------------------------------------------------------
%	SECTION 3
%----------------------------------------------------------------------------------------

\section{Design (D)}

\subsection*{D1(i, -)}

Multiple courses have developed and contributed to my understanding and ability to evaluate business, customer and user needs:
\textit{Introduction to Design},
\textit{Introduction to the Environment},
\textit{Environment and Behaviour},
\textit{Critical Architectural Studies},
\textit{Electrical and Lighting Services for Buildings},
\textit{Procurement and Contracts},
\textit{Facilities Management Principles},
\textit{Inclusive and Safe Environments},
and \textit{Design Project}.
The concept of a good brief that reflects the business, customer and user needs has been emphasised throughout the AE programme.
I have been able to evaluate such needs through various assignments and in different contexts, such as technical aspects (air quality, lighting levels and fire safety), users with different kinds of disabilities, and public perception.
During my placement at Hoare Lea, I was able to evaluate the design brief of a mixed-use new-build for needs regarding fire safety in order to design the fire detection and alarm systems.

%Some of these courses considered more technical needs such as good air quality and lighting levels, which are dependent on the functions of spaces.
%In \textit{Procurement and Contracts} and \textit{Facilities Management Principles}, I learned about the different stakeholders involved in a project and their respective needs and interests.


\subsection*{D2(i, -)}

\hl{Investigating and defining problems...}


\subsection*{D3(-, b, m)}

Working with information that may be incomplete or uncertain is something that I have found difficult.
\textit{Design Project B}, \textit{Design Project} and \textit{Laboratory Project} are courses that particularly challenged me in this area and helped me develop the skill.
During these projects I learned to make assumptions and use iterative/ trial-and-error processes to come up with a satisfactory design solution.
This was done, for example, for the sizing of pipes in the design projects.
\hl{However, I do not think I have ever gone as far as to quantify the effect of incomplete or uncertain information on a design and to use research to mitigate the deficiencies.}


\subsection*{D4(i, -)}

All design projects have developed my skills in creating design solutions that are fit for purpose:
\textit{1st year collaborative project},
\textit{Design Project A},
\textit{Design Project B},
\textit{Critical Architectural Studies},
\textit{Energy and Buildings},
and \textit{Design Project}, including the \textit{4th year collaborative project}.
\hl{Give examples?!
However, I cannot think of an example where the whole lifecycle was considered, especially disposal.}


\subsection*{D5(i, -)}

\hl{Plan and manage the design process, including cost drivers...
Design Project? I didn't do the timeline or costs...
I have always kind of avoided doing costing (CAS, collab projects (for appropriate reason)).}


\subsection*{D6}

\hl{Just in the context of design?}

Most of the assignments I have written and presentations I have given contributed to the development of my skill of communicating to technical or non-technical audiences.
Courses and work placements that have contributed to this skill development include:
\textit{Design Project A},
\textit{Construction Technology 2},
\textit{Hydraulics and Hydrology A},
\textit{Environment and Behaviour},
\textit{Design Project B},
\textit{Critical Architectural Studies},
\textit{Thermal Performance Studies},
\textit{Design Issues},
\textit{Facilities Management Principles},
\textit{Inclusive and Safe Environments},
\textit{Laboratory Project},
\textit{Design Project},
\textit{Dissertation},
\textit{Climate Change, Sustainability and Adaptation},
Sunamp,
Sweco?
Instances of communicating to non-technical audiences include my POSTnote assignment for \textit{Climate Change, Sustainability and Adaptation}, where I wrote about a technical topic to members of the UK parliament, and the Product Information Sheets that I created for Sunamp, which needed to convey the technical operation of their products in layman's terms.
Most laboratory reports I have written, e.g. for \textit{Laboratory Project}, were of a technical nature.


\subsection*{D7m}

I have been exposed to various UK building design and construction processes (especially the Royal Institute of British Architects (RIBA) Plan of Work 2013) and procurement routes during the following courses and work placements:
Arup,
\textit{Procurement and Contracts},
\textit{Facilities Management Principles},
Hoare Lea,
and
\textit{Dissertation}.
I have been able to contrast this knowledge with my (albeit limited) knowledge of Swedish building design and construction processes gained during my placement at Hultin \& Lundquist.
The thermofluids part of \textit{Laboratory Project} taught me about the methodology for optimising a design.
This knowledge has been complemented by information about specific \hl{parts/ scopes} of the design process, such as the briefing stage, site analysis, POEs and life cycle assessments, learned in
\textit{Construction Technology 1},
\textit{Introduction to Design},
\textit{Environment and Behaviour},
\textit{Facilities Management Principles},
and \textit{Sustainable and Intelligent Buildings}.
I have also delved into a study of BIM, a new way to approach collaborative building design which is the new ``Holy Grail" of design processes, during my \textit{Dissertation} and \textit{Innovation in Construction Practice}.
%UK vs Sweden: Arup, 
%HL, FMP, DST, P\&C vs H\&L
%BIM: DST, ICP
%Engineering optimisation: LAB
%From site analysis to POEs + LCA: Con tech 1, Intro to Design, Env and Beh, SIB
All of these courses and experiences have contributed to my wide knowledge and comprehensive understanding of design processes and methodologies.

I have developed my ability to apply and adapt this wide repository of knowledge and understanding of design processes and methodologies in the unfamiliar situations presented in \textit{Design Project} and \textit{Critical Architectural Studies}.
\hl{Elaborate?}
%Application in unfamiliar situations: DP, CAS


\subsection*{D8m}

\hl{New needs?}


%----------------------------------------------------------------------------------------
%	SECTION 4
%----------------------------------------------------------------------------------------

\section{Economic, Legal, Social, Ethical and Environmental Context (EL)}


%----------------------------------------------------------------------------------------
%	SECTION 5
%----------------------------------------------------------------------------------------

\section{Engineering Practice (P)}


%----------------------------------------------------------------------------------------
%	SECTION 6
%----------------------------------------------------------------------------------------

\section{Additional General Skills (G)}