\chapter{Skills Analysis} % Main chapter title

\label{Chapter3} % Change X to a consecutive number; for referencing this chapter elsewhere, use \ref{Chapter3}

\lhead{Chapter 3. \emph{Skills Analysis}} % Change X to a consecutive number; this is for the header on each page - perhaps a shortened title

In this chapter I describe, analyse and demonstrate the knowledge and skills I have developed and gained throughout my university education
(see courses in Table~\ref{tbl:courses})
and work experiences in industry
(see placements in Figure~\ref{timeline}).
This has been done in line with the learning outcomes (LOs) stipulated by the Engineering Council for the university programmes that the council accredits (see Appendix~\ref{App:ECLOs}).
%, as well as the Graduate Attributes of Heriot-Watt University (see Appendix~\ref{App:HWgrad}).


% Please add the following required packages to your document preamble:
% \usepackage{booktabs}
% \usepackage{multirow}
\begin{table}[htbp]
	\caption{The courses I took for MEng Architectural Engineering}
	\label{tbl:courses}
	\centering
	\begin{tabular}{@{}lll@{}}
		\toprule
		\textbf{} & \textbf{Code} & \textbf{Course Title} \\ \midrule
		\multirow{8}{*}{\textbf{Year 1}} & D37TA & Construction Technology 1 \\
		& D17HH & History of the Built Environment \\
		& D17DE & Introduction to Design \\
		& F17XA & Mathematics for Engineers and Scientists 1 \\
		& D18BT & Building Services Technology \\
		& D27MB & Mechanics B \\
		& D47IE & Introduction to the Environment \\
		& F17XB & Mathematics for Engineers and Scientists 2 \\ \midrule
		\multirow{8}{*}{\textbf{Year 2}} & D18PA & Design Project A \\
		& D38TA & Construction Technology 2 \\
		& D18AB & Acoustics and Architectural Design \\
		& D28HA & Hydraulics and Hydrology A* \\
		& D17BG & Environment and Behaviour \\
		& F78SC & Statistics for Science \\
		& D18EP & Energy Principles and Applications \\
		& D18DP & Design Project B* \\ \midrule
		\multirow{8}{*}{\textbf{Year 3}} & D19CX & Critical Architectural Studies \\
		& D19EL & Electrical and Lighting Services for Buildings \\
		& D19SO & Design Software Applications \\
		& D39PZ & Procurement and Contracts \\
		& D19EB & Energy and Buildings \\
		& D19TP & Thermal Performance Studies \\
		& D19DI & Design Issues \\
		& D38FM & Facilities Management Principles \\ \midrule
		\multirow{6}{*}{\textbf{Year 4}} & D10YD/F & Design Project (AE) (S1/S2) \\
		& D10ZA/B & Dissertation (AE) (S1/S2) \\
		& D10LP & Laboratory Project \\
		& D10IE & Inclusive and Safe Environments \\
		& D10IS & Sustainable and Intelligent Buildings \\
		& D30IC & Innovation in Construction Practice \\ \midrule
		\multirow{7}{*}{\textbf{Year 5}} & D11PJ & Industrial Project \\
		& D11CA & Climate Change, Sustainability and Adaptation* \\
		& D21WC & Water Supply and Drainage for Buildings* \\
		& \textcolor{gray}{D11AF} & \textcolor{gray}{Architectural Acoustics} \\
		& \textcolor{gray}{D11DC} & \textcolor{gray}{Design of Low Carbon Buildings} \\
		& \textcolor{gray}{D11TH} & \textcolor{gray}{Thermofluids} \\
		& \textcolor{gray}{--} & \textcolor{gray}{\textit{To be determined}*} \\ \bottomrule
		\multicolumn{3}{l}{\textit{*Optional courses}} \\
		\multicolumn{3}{l}{\textcolor{gray}{\textit{Courses to be taken in 2019 are displayed in grey}}} \\
	\end{tabular}
\end{table}


I am studying an integrated Master of Engineering (MEng) degree in Architectural Engineering at Heriot-Watt University.
The degree is accredited by the Chartered Institution of Building Services Engineers (CIBSE), which is a licensed institution of the Engineering Council.
This means that when somebody joins CIBSE, they will also be registered as a Chartered Engineer (CEng), Incorporated Engineer (IEng) or Engineering Technician (EngTech) when they have reached the appropriate level of qualification and professional skill \citep{whyjoinCIBSE}.

The Engineering Council has 42 LOs for the degree programmes it accredits.
These are divided up into six categories:
\begin{itemize}
    \item Science and mathematics (SM)
    \item Engineering analysis (EA)
    \item Design (D)
    \item Economic, legal, social, ethical and environmental context (EL)
    \item Engineering practice (P)
    \item Additional general skills (G)
\end{itemize}
Each category has a list of LOs which are depicted by the category abbreviation (e.g. SM) and a number (e.g. 1); examples of LOs are SM1, EA2 and D3.
Most of these LOs are divided into, at the most, three incrementally increasing attainment levels which are depicted by the letters ``i", ``b" or ``m", e.g. SM1i, SM1b and SM1m.
From the \textit{UK Standard for Professional Engineering Competence} (UK-SPEC) \citep{EngineeringCouncil2014}, 
I have been able to deduce that:
\begin{itemize}
    \item \textbf{``i"} represents the LOs of Bachelors Degrees and Bachelors (Honours) Degrees accredited for IEng registration.
    \item \textbf{``b"} represents the LOs of Bachelors (Honours) Degrees accredited as partially meeting the educational requirement for CEng registration.
    \item \textbf{``m"} represents the LOs of integrated Masters (MEng) degrees accredited for CEng registration.
\end{itemize}

Some LOs are only divided into two attainment levels, one of which is not depicted by a letter ``i", ``b" or ``m", e.g. EA2i and EA2.
In such cases, the LO with the letter ``i", ``b" or ``m" applies to the corresponding attainment level and the LO without the letter applies to all the other attainment levels.
In the example of EA2i and EA2, EA2 applies to both attainment levels ``b" and ``m".

Other LOs, however, are not depicted by any of the letters ``i", ``b" or ``m", e.g. D6.
These LOs therefore apply to all three of the attainment levels.\footnote{
For consistency, I have amended seven of the LO abbreviations given to us in the D11PJ Course Handbook.
The LOs D3, EL3, EL6, P2, P11 and G3 were respectively changed to D3i, EL3b, EL6b, P2b, P11b and G3b.
Since the LOs P4 and P4m are identical, I have merged them to the single LO P4.
These amendments were done so that the naming conventions described above always apply.}


%I have developed a skills matrix (see Appendix~\ref{App:matrix}) in which I show the courses and placements where I have developed each LO attainment level.
I have developed a skills matrix (see Appendix~\ref{App:matrix}) in which I show the extent to which each course or work placement has contributed to my achievement of each LO.
I have used three symbols to depict my level of achievement
and have attributed a numerical value to each (see Table~\ref{tbl:symbols}).


\begin{table}[htbp]
	\caption{The achievement level symbols used in this report.}
	\label{tbl:symbols}
	\centering
	\begin{tabular}{@{}lccc@{}}
		\toprule
		\textbf{Symbol} & \littlemaster & \nomaster & \master \\ \midrule
		\textbf{Achievement level} & Limited & Above average & Full \\
		\textbf{Numerical value} & 1 & 2 & 3 \\ \bottomrule
	\end{tabular}
\end{table}



\begin{wraptable}[5]{r}{0.3\textwidth}
	\centering
	\caption{Example box}
	\label{tbl:example_box}
	\begin{tabular}{|c|}
		\hline
		\rowcolor[HTML]{F8A102} 
		\multicolumn{1}{|c|}{\cellcolor[HTML]{F8A102}\textbf{LO(i, b, m) \littlemaster / \nomaster / \master}} \\ \hline
		Course codes \\
		Work placements \\ \hline
	\end{tabular}
\end{wraptable}

The approach of my skills analysis in this chapter is competency-based.
I have grouped the attainment levels of the LOs, such as SM1(i, b, m) and EA2(i, -), where the dash ``-" depicts the LO without ``i", ``b" or ``m" as an ending letter.
Each LO is presented with a box, like the one in Table \ref{tbl:example_box}.
The box lists the codes of the courses and the companies I have worked at that helped me achieve that LO.
(I sometimes abbreviate the company name ``Hultin \& Lundquist Arkitekter" to ``H\&L" to make it fit neatly in the boxes.)
An achievement level symbol will show my overall achievement of the LO, considering all of its attainment levels (i, b and m).
If there is no symbol in the box, this means that I have not achieved that LO.

The UK-SPEC uses the following terms and their definitions, which will also apply within this report \citep{EngineeringCouncil2014}:
%\hl{Re-arranged extract from AHEP 3rd edition}:
\begin{itemize}
    \item \textbf{Awareness} is general familiarity, albeit bounded by the needs of the specific discipline

    \item \textbf{Knowledge} is information that can be recalled

    \item \textbf{Understanding} is the capacity to use concepts creatively, for example, in problem solving, design, explanations and diagnosis

    \item \textbf{Skills} are acquired and learned attributes that can be applied almost automatically

    \item \textbf{Know-how} is the ability to apply learned knowledge and skills to perform operations intuitively, efficiently and correctly

    \item \textbf{Complex} implies engineering problems, artefacts or systems that involve dealing simultaneously with a sizeable number of factors that interact and require deep understanding, including knowledge at the forefront of the discipline, to analyse or deal with

\end{itemize}





%----------------------------------------------------------------------------------------
%	SECTIONS
%----------------------------------------------------------------------------------------

%----------------------------------------------------------------------------------------
%	SECTION 1
%----------------------------------------------------------------------------------------

\section{Science and Mathematics (SM)}

\subsection*{SM1(i, b, m)}

Multiple courses (mainly from years 2 and 3) have contributed to my knowledge and understanding of scientific principles and methodology necessary to underpin my education in Architectural Engineering.
A main contributor was \textit{Thermal Performance Studies}.
On this course I learned the scientific principles of psychrometrics, i.e. the properties of air-vapour mixtures (especially/ like air), and how to use this knowledge in the design of air conditioners, heat exchangers and cooling towers.
Learning about the environmental effects of refrigerants has contributed to my understanding of the evolution of air conditioners, from using carbon- and fluoride-based refrigerants to water and eventually phase change materials (PCMs) like Sunamp uses.

Learning the scientific principles of related disciplines has helped me enrich my appreciation of the scientific and engineering context of my discipline.
\textit{Statistics for Science} has, for example, helped me understand the proper methodology behind sampling populations and to identify bias in data.
\textit{Hydraulics and Hydrology A}, a civil engineering course, helped me appreciate the hydrological cycle (relevant to rainwater and urban drainage) and water flow in pipes (relevant to drainage, water supply and heating systems).


\subsection*{SM2(i, b, m)}

\textit{Mathematics for Engineers and Scientists 1 and 2} and \textit{Statistics for Science} all further developed my knowledge and understanding of mathematical and statistical methods.
I have been able to apply this knowledge in the analysis and solution of engineering problems.
For example, I used it to work out the yield of an array of photovoltaic panels (PV) in \textit{Energy and Buildings} as well as the yield of a set of turbines in a tidal energy generation system as part of my group project in \textit{Design Project}.


\subsection*{SM3(b, m)}

Other engineering disciplines that I have gained knowledge and understanding from are predominantly civil and structural.
These were acquired through the following courses: \textit{Mechanics B}, \textit{Construction Technology 1 and 2},  and \textit{Hydraulics and Hydrology A}.
%\hl{I cannot remember how I have applied or integrated this knowledge though...
%pipes?
%materials choice?}
I have been able to apply and integrate this knowledge and understanding to support study of my own engineering discipline, Architectural Engineering.
For example, during a \hl{CFD (1st time I mention CFD?)} modelling exercise in Laboratory Project, I was able to identify laminar and turbulent flows, something I had learned about it in \textit{Hydraulics and Hydrology A}.
This enabled me to refine and improve my CFD model.


\subsection*{SM4m}

A range of courses increased my awareness of developing technologies relevant to Architectural Engineering and building services.
In \textit{Building Services Technology}, \textit{Energy and Buildings}, \textit{Dissertation} and \textit{Innovation in Construction Practice}, I respectively learned about technologies such as modern windcatchers, smart meters, Building Information Modelling (BIM) and virtual and augmented reality.
I was also exposed to a developing technology during my placement at Sunamp: their heat batteries.
\hl{refer to work done at Sunamp?}


\subsection*{SM5m}

\hl{Is this skill about software???}
\textit{Design Software Applications} and \textit{Laboratory Project} enabled me to gain a comprehensive knowledge and understanding of software used by architectural engineers.
I learned how steady-state and dynamic software applications function, was made aware of the mathematical \hl{models/ formulas} they are based on and learned about their limitations.
\hl{Should I give examples of limitations to demonstrate my knowledge/ understanding? Not necessarily part of marking criteria.}
I was introduced to and got to practise using the following applications:
\begin{itemize}
    \item Steady-state: iSBEM (an interface for Simplified Building Energy Model), SAP (Standard Assessment Procedure) and PHOENICS (a CFD software application)
    \item Dynamic: IES-VE (Integrated Environmental Solutions - Virtual Environment)
    \item Computational fluid dynamics (CFD): PHOENICS?
\end{itemize}

% Please add the following required packages to your document preamble:
% \usepackage{booktabs}
\begin{table}[htbp]
\begin{tabular}{@{}lp{8cm}@{}}
\toprule
Type of software & Applications \\ \midrule
Steady-state & iSBEM (an interface for Simplified Building Energy Model) \\
 & SAP (Standard Assessment Procedure) \\
 & PHOENICS (a CFD software application) \\
Dynamic & IES-VE (Integrated Environmental Solutions - Virtual Environment) \\
Computational fluid dynamics (CFD)? & PHOENICS? \\ \bottomrule
\end{tabular}
\end{table}

\hl{List or table?}

\hl{Although it is outside the scope of this learning outcome, I would like to add that these courses did not allow me to properly develop my skills in using the software...}


\subsection*{SM6m}

\hl{How does this fall under Science and Maths heading?}

I have been able to critically evaluate and apply a range of concepts, including some outside engineering, in my engineering projects.
\hl{EXAMPLE 1:}
I learned about the sustainable design hierarchy (see Figure \ref{fig_hierarchy}) during my placement at Arup.
I then applied this approach to the design of a library for a group project in \textit{Critical Architectural Studies}, helping us earn the first prize in Sustainable Design (see Figure \ref{fig_award}).
\hl{EXAMPLE 2:}
Upon conducting a post-occupancy evaluation (POE) for \textit{Environment and Behaviour}, I came across a social sciences journal paper about survey research.
It analysed typical response styles and suggested how to design questionnaires to avoid biased or inaccurate data collection.
Using these \hl{criteria/ suggestions}, I was able to critically evaluate the success of the questionnaire used for the POE in \textit{Environment and Behaviour} and design an improved survey questionnaire for the acoustic \textit{Laboratory Project}.

\begin{figure}[htbp]
	\centering
	\includegraphics[width=10cm]{figures/Hierarchy.jpg}
	\rule{\textwidth}{0.5pt} % use line???
	\caption{The sustainable design hierarchy %\citep{Dougherty:online}.
	}
	\label{fig_hierarchy}
\end{figure}


\begin{figure}[htbp]
	\centering
	\includegraphics[width=\textwidth]{figures/SustainabilityAward.png}
	\rule{\textwidth}{0.5pt} % use line???
	\caption{First prize for sustainability awarded to my group for our library design in \textit{Critical Architectural Studies} (D19CX).}
	\label{fig_award}
\end{figure}

\hl{Other courses?}
\begin{itemize}
	\item Use of solar gain in pool house (\textit{first year collaborative project})?
	\item Contact factor in \textit{LAB CFD}?
	\item Learned in \textit{CAS}, applied again in \textit{y4 collab}: make design concept red thread...
\end{itemize}



%----------------------------------------------------------------------------------------
%	SECTION 2
%----------------------------------------------------------------------------------------

\section{Engineering Analysis (EA)} \label{EA}

\subsection*{EA1(i, b, m)} 

\begin{wraptable}{r}{0.2\textwidth}
	\begin{tabular}{|ll|}
		\hline
		\multicolumn{2}{|c|}{\cellcolor[HTML]{F8A102}\textbf{EA1(i, b, m) \nomaster}} \\ \hline
		\EPA & \DSA \\
		\TPS & \LAB \\ \hline
	\end{tabular}
\end{wraptable}

In some courses, I have conducted laboratory experiments and modelling exercises.
This has further developed my skills in monitoring, interpreting and applying the results of analysis and modelling to bring about continuous improvement [EA1i].
\textit{Laboratory Project} went a step further and taught me the principles and process to engineering a better solution [EA1b and EA1m].
These principles were applied in an engineering project which aimed to optimise the design of a cooling coil.
Having earned As on all of the courses listed in the course box shows that I have an understanding of engineering principles and the ability to apply them to analyse key engineering processes.








\subsection*{EA2(i, -)}

\begin{wraptable}{r}{0.2\textwidth}
	\begin{tabular}{|ll|}
		\hline
		\multicolumn{2}{|c|}{\cellcolor[HTML]{F8A102}\textbf{EA2(i, -) \nomaster}} \\ \hline
		\ConTechOne & \Acoustics \\
		\HYD & \EPA \\
		\TPS & \PRJ \\
		\LAB &  \\
		Arup & Sweco \\ \hline
	\end{tabular}
\end{wraptable}

A range of courses have developed my ability to understand and explain the performance of systems and components through the use of quantitative/ analytical methods and/ or modelling techniques.
For example, in \textit{Hydraulics and Hydrology A}, I learned how to use quantitative methods to distinguish laminar flows from turbulent flows.
And when we were studying thermofluids in \textit{Laboratory Project}, I used both CFD modelling and analytical methods to understand and describe the performance of a cooling coil in an air conditioning unit.

I have also applied quantitative/ analytical methods and/ or modelling techniques to classify and describe the performance of systems during my work placements at Arup and Sweco.
In particular, at Sweco I used a combination of the aforementioned methods and techniques to describe the performance of some buildings in terms of solar heat gain, energy consumption, thermal comfort, and daylighting.
The results of these analyses were then classified according to the criteria of \textit{Miljöbyggnad}: Gold, Silver, Bronze or Disapproved (see Figure~\ref{fig:svl}).








\subsection*{EA3(i, b, m)} \label{EA3}

\begin{wraptable}{r}{0.2\textwidth}
	\begin{tabular}{|ll|}
		\hline
		\multicolumn{2}{|c|}{\cellcolor[HTML]{F8A102}\textbf{EA3(i, b, m) \nomaster}} \\ \hline
		\MechB & \HYD \\
		\DPB & \DSA \\
		\EnBldgs & \TPS \\
		\PRJ & \LAB \\ \hline
	\end{tabular}
\end{wraptable}

I have developed the ability to use the results of analysis and apply quantitative and computational methods to solve engineering problems and subsequently recommend or implement appropriate action in some courses.
For example, a combination of calculations, CFD modelling and a physical laboratory experiment were used to optimise the design of a cooling coil in \textit{Laboratory Project}.
This was an iterative process of solving problems and implementing appropriate actions to come up with a better solution, for which I earned an A.
However, I do not think I have ever used alternative approaches to solve engineering problems and implement appropriate action [EA3m]; this is a skill I need to develop.







\subsection*{EA4(i, b, m)}

\begin{wraptable}{r}{0.2\textwidth}
	\begin{tabular}{|ll|}
		\hline
		\multicolumn{2}{|c|}{\cellcolor[HTML]{F8A102}\textbf{EA4(i, b, m) \nomaster}} \\ \hline
		\CAS & \EnBldgs \\
		\PRJ & \WSD \\ \hline
	\end{tabular}
\end{wraptable}

Some courses have contributed to my understanding of, and ability to apply, an integrated or systems approach to solving complex engineering problems through know-how of the relevant technologies and their application.
I would not say this skill is fully developed though as I have not had many opportunities to develop it.
An example of when I did fully demonstrate my ability to achieve LOs EA4i, EA4b and EA4m is my natural lighting design of Newington Library in \CASTitle.
With the aim of maximising daylight and avoiding direct sunlight, I had to take several aspects into consideration in the fenestration design:
\begin{itemize}
	\item The building orientation and sun path
	\item The characteristics and desirability of the natural light hitting each fa{\c{c}}ade
	\item Functions of the internal and external spaces and their required lighting levels
	\item Type of glazing (double/ triple, clear/ frosted)
	\item Subsequent thermal comfort and energy loads related to the choice of glazing
\end{itemize}


I will give an example to demonstrate the interdependent nature of this design exercise.
The south-west corner of the library housed a shelving and study area.
However, the south-west fa{\c{c}}ade would be receiving an intense afternoon sunlight, which was undesirable due to glare and the sunlight's detrimental effects on the books.
I therefore designed clerestory glazing to (a) limit direct incoming sunlight, (b) exclude distracting, unattractive or intrusive views, and (c) allow shelving against wall (see Figure~\ref{fig:nl}).
Additionally, the clerestory glazing was frosted to disperse the sunlight's rays.
My natural lighting design was commended by the judges of the course's design competition (see Figure~\ref{fig_award}).


\begin{figure}[htbp]
	\centering
	\includegraphics[width=\textwidth]{figures/NL-nl.png}
	\rule{\textwidth}{0.5pt} % use line???
	\caption{The fit-for-purpose glazing scheme of Newington Library.}
	\label{fig:nl}
\end{figure}

%\hl{If I look overll at all the steps (integrated), I can go down that path. but if I just look at the step that isn't working, I'd go down another path. Look at something that wasn't working, and . CCSA eposter: not to miss point of question/ topic (how to harness lessons from past climate events); focusing on other parts of question sets us up to fail. Because I know about this, I can apply it to engineering in the following ways. Sunamp PISs. Made sure whole process was more carefully thought out than it was when I was given the job. I had an overview of the whole thing. I made sure that the document enabled people to use the batteries as efficiently as possible. Can't do that by looking at each point of its own - it's got to flow as a document. specific application of heat batteries  - know-how  of technology and application (0.5 page)}









\subsection*{EA5m}

\begin{wraptable}{r}{0.2\textwidth}
	\begin{tabular}{|ll|}
		\hline
		\multicolumn{2}{|c|}{\cellcolor[HTML]{F8A102}\textbf{EA5m \master}} \\ \hline
		\EnBldgs & \PRJ \\
		\DST & \ICP \\
		Hoare Lea & Sunamp \\ \hline
	\end{tabular}
\end{wraptable}

Some courses and placements have enabled me to develop my ability to use fundamental knowledge to investigate new and emerging technologies.
My most extensive investigation is of Sunamp's heat batteries, having developed the PISs and a Product Selection Quiz for their UniQ product range (see Appendices~\ref{App:PISs} and \ref{App:quiz}).
My knowledge of heat exchangers (gained notably in the thermofluids part of the \LABTitle),
water supply and heating (gained in the \PRJTitle),
and renewable technologies (gained in \EnBldgsTitle)
contributed to my understanding of the operation of Sunamp's heat batteries.










\subsection*{EA6m} \label{sec:EA6m}

\begin{wraptable}{r}{0.2\textwidth}
	\begin{tabular}{|ll|}
		\hline
		\multicolumn{2}{|c|}{\cellcolor[HTML]{F8A102}\textbf{EA6m \littlemaster}} \\ \hline
		\multicolumn{2}{|c|}{\PRJ} \\ \hline
	\end{tabular}
\end{wraptable}

I do not have a lot of experience extracting and evaluating pertinent data to apply engineering analysis techniques in the solution of unfamiliar problems.
The only example I can think of is when I tried to calculate the yield of tidal power for the \textit{4th year collaborative project}, which was part of the \PRJTitle.
This was an unfamiliar problem and I could only find information online on how to calculate yields from hydroelectric dams.
Thus, I had to use a combination of the information I found, assumptions and my own fundamental mathematical skill set.
I found that four bi-directional turbines with a total of four ebb and flood tides of three metres could generate a daily total of 284~kWh, but this was only 16\% of the building's daily electricity demand.
Since I never showed my workings to anyone, I cannot be sure if they were correct.
Overall, I believe I could use some more development in this skill area.

%----------------------------------------------------------------------------------------
%	SECTION 3
%----------------------------------------------------------------------------------------

\section{Design (D)}

\subsection*{D1(i, -)}

Multiple courses have developed and contributed to my understanding and ability to evaluate business, customer and user needs:
\textit{Introduction to Design},
\textit{Introduction to the Environment},
\textit{Environment and Behaviour},
\textit{Critical Architectural Studies},
\textit{Electrical and Lighting Services for Buildings},
\textit{Procurement and Contracts},
\textit{Facilities Management Principles},
\textit{Inclusive and Safe Environments},
and \textit{Design Project}.
The concept of a good brief that reflects the business, customer and user needs has been emphasised throughout the AE programme.
I have been able to evaluate such needs through various assignments and in different contexts, such as technical aspects (air quality, lighting levels and fire safety), users with different kinds of disabilities, and public perception.
During my placement at Hoare Lea, I was able to evaluate the design brief of a mixed-use new-build for needs regarding fire safety in order to design the fire detection and alarm systems.

%Some of these courses considered more technical needs such as good air quality and lighting levels, which are dependent on the functions of spaces.
%In \textit{Procurement and Contracts} and \textit{Facilities Management Principles}, I learned about the different stakeholders involved in a project and their respective needs and interests.


\subsection*{D2(i, -)}

\hl{Investigating and defining problems...}


\subsection*{D3(i, b, m)}

Working with information that may be incomplete or uncertain is something that I have found difficult.
\textit{Design Project B}, \textit{Design Project} and \textit{Laboratory Project} are courses that particularly challenged me in this area and helped me develop the skill.
During these projects I learned to make assumptions and use iterative/ trial-and-error processes to come up with a satisfactory design solution.
This was done, for example, for the sizing of pipes in the design projects.
\hl{However, I do not think I have ever gone as far as to quantify the effect of incomplete or uncertain information on a design and to use research to mitigate the deficiencies.}

\hl{See notes on Notion print-out from 26 Oct 2018}


\subsection*{D4(i, -)}

All design projects have developed my skills in creating design solutions that are fit for purpose:
\textit{1st year collaborative project},
\textit{Design Project A},
\textit{Design Project B},
\textit{Critical Architectural Studies},
\textit{Energy and Buildings},
and \textit{Design Project}, including the \textit{4th year collaborative project}.
\hl{Give examples?! However, I cannot think of an example where the whole lifecycle was considered, especially disposal.}




\subsection*{D5(i, -)}

\begin{wraptable}[5]{r}{0.2\textwidth}
	\begin{tabular}{|ll|}
		\hline
		\multicolumn{2}{|c|}{\cellcolor[HTML]{F8A102}\textbf{D5(i, -)}} \\ \hline
		\IE & \DPA \\
		\DPB & \CAS \\
		\PRJ & \DST \\
		\LAB & \CCSA \\ \hline
	\end{tabular}
\end{wraptable}

I have had some experience in planning and managing design processes, notably in group design projects (whether it was the design of a building or a poster \ldots).
I have also evaluated the outcomes of a design process, notably in a reflective report submitted after the design of a library in \CASTitle.
Some reflective points from that report (on which I got an A$+$) on what makes a good design process are \citep{eklowCAS}:
\begin{itemize}
	\item ``a strong concept [\ldots] helped guide us through all of our decision-making and eventually come up with a good-quality design"
	\item ``The final group design could have been improved [\ldots] if the building solution had been finalised earlier [\ldots]. This could have been achieved by planning the design timeline right from the start, something we failed to do."
	\item ``I learned that tasks should be sized and assigned according to each individual’s commitment/ reliability" rather than in fairness.
\end{itemize}

I have, however, never planned or managed the cost drivers of a design process, despite having had opportunities to do this.
The task of costing was either allocated to another group member, or (as was the case for the \PRJTitle) I failed to do it due to poor time management.







\subsection*{D6}

\hl{Just in the context of design?}

Most of the assignments I have written and presentations I have given contributed to the development of my skill of communicating to technical or non-technical audiences.
Courses and work placements that have contributed to this skill development include:
\textit{Design Project A},
\textit{Construction Technology 2},
\textit{Hydraulics and Hydrology A},
\textit{Environment and Behaviour},
\textit{Design Project B},
\textit{Critical Architectural Studies},
\textit{Thermal Performance Studies},
\textit{Design Issues},
\textit{Facilities Management Principles},
\textit{Inclusive and Safe Environments},
\textit{Laboratory Project},
\textit{Design Project},
\textit{Dissertation},
\textit{Climate Change, Sustainability and Adaptation},
Sunamp,
Sweco?
Instances of communicating to non-technical audiences include my POSTnote assignment for \textit{Climate Change, Sustainability and Adaptation}, where I wrote about a technical topic to members of the UK parliament, and the Product Information Sheets that I created for Sunamp, which needed to convey the technical operation of their products in layman's terms.
Most laboratory reports I have written, e.g. for \textit{Laboratory Project}, were of a technical nature.


\subsection*{D7m}

I have been exposed to various UK building design and construction processes (especially the Royal Institute of British Architects (RIBA) Plan of Work 2013) and procurement routes during the following courses and work placements:
Arup,
\textit{Procurement and Contracts},
\textit{Facilities Management Principles},
Hoare Lea,
and
\textit{Dissertation}.
I have been able to contrast this knowledge with my (albeit limited) knowledge of Swedish building design and construction processes gained during my placement at Hultin \& Lundquist.
The thermofluids part of \textit{Laboratory Project} taught me about the methodology for optimising a design.
This knowledge has been complemented by information about specific \hl{parts/ scopes} of the design process, such as the briefing stage, site analysis, POEs and life cycle assessments, learned in
\textit{Construction Technology 1},
\textit{Introduction to Design},
\textit{Environment and Behaviour},
\textit{Facilities Management Principles},
and \textit{Sustainable and Intelligent Buildings}.
I have also delved into a study of BIM, a new way to approach collaborative building design which is the new ``Holy Grail" of design processes, during my \textit{Dissertation} and \textit{Innovation in Construction Practice}.
%UK vs Sweden: Arup, 
%HL, FMP, DST, P\&C vs H\&L
%BIM: DST, ICP
%Engineering optimisation: LAB
%From site analysis to POEs + LCA: Con tech 1, Intro to Design, Env and Beh, SIB
All of these courses and experiences have contributed to my wide knowledge and comprehensive understanding of design processes and methodologies.

I have developed my ability to apply and adapt this wide repository of knowledge and understanding of design processes and methodologies in the unfamiliar situations presented in \textit{Design Project} and \textit{Critical Architectural Studies}.
\hl{Elaborate?}
%Application in unfamiliar situations: DP, CAS


\subsection*{D8m}

\begin{wraptable}{r}{0.2\textwidth}
	\begin{tabular}{|ll|}
		\hline
		\multicolumn{2}{|c|}{\cellcolor[HTML]{F8A102}\textbf{D8m}} \\ \hline
		\PRJ & Sunamp \\ \hline
	\end{tabular}
\end{wraptable}

The way that I developed the UniQ Product Selection Quiz (described in Section~\ref{sec:quiz}) is a demonstration of my ability to generate an innovative design for products, systems, components or processes to fulfil new needs.
I also came up with the innovative idea of generating tidal power for my 4\textsuperscript{th} year collaborative project (described in Section~\ref{sec:P10m}), but this was not fully designed.
Since I have not created many innovative designs to fulfil new needs, I would not say that I have fully developed or mastered this skill.
%----------------------------------------------------------------------------------------
%	SECTION 4
%----------------------------------------------------------------------------------------

\section{Economic, Legal, Social, Ethical and Environmental Context (EL)}

\subsection*{EL1(-, m)}

\begin{wraptable}{r}{0.2\textwidth}
	\begin{tabular}{|ll|}
        \hline
        \multicolumn{2}{|c|}{\cellcolor[HTML]{F8A102}\textbf{EL1(-, m) \nomaster}} \\ \hline
        \PC & \DST \\
        \LAB & \IP \\
        Arup & Sunamp \\
        Sweco &  \\ \hline
	\end{tabular}
\end{wraptable}

I gained some understanding of the need for a high level of professional and ethical conduct in engineering and some knowledge of professional codes of conduct in \textit{Procurement and Contracts}.
I familiarised myself with CIBSE's and the Engineering Council's professional codes of conduct in an assignment where I advised a client how to redress the unethical behaviours in a \pounds35~million new railway station development.
One of the unethical issues addressed was the main consultant's conflict of interest, as he was simultaneously working for another person with competing interests on the project.
By referring to various clauses of CIBSE's professional code of conduct, I advised the client to first send a warning letter to the consultant, and if the consultant repeated their offence, to consider sending a formal report to CIBSE.
If the latter course of action were to be taken, I made the client aware that CIBSE's disciplinary actions may compromise the consultant and negatively impact the project.
My awarded mark of an A on this assignment demonstrates my understanding of LOs EL1 and EL1m.

Since then, I have practised my professional and ethical conduct in the collection and analysis of surveys (which had to be kept confidential) and even during the writing of this report.
At Sunamp, I signed a confidentiality agreement.
Thus, I sent Susan (COO) drafts of this report to ensure I did not unintentionally reveal any confidential information about Sunamp.

I have, however, not revisited the professional codes of conduct since \PCTitle.
This leaves me with an overall partial achievement of LOs EL1 and EL1m.




\subsection*{EL2}

\begin{wraptable}{r}{0.2\textwidth}
	\begin{tabular}{|ll|}
		\hline
		\multicolumn{2}{|c|}{\cellcolor[HTML]{F8A102}\textbf{EL2} \nomaster} \\ \hline
		\ID & \DI \\
		\DST & \SIB \\
		Sunamp &  \\ \hline
	\end{tabular}
\end{wraptable}


I have gained knowledge and understanding of the commercial, economic and social context of engineering processes.
A pertinent example which touches all of these aspects is one that I gave in a debate in \IDTitle \space after having heard about it from Stuart MacPherson, the owner of Irons Foulner Consulting Engineers.
Around 2002, Stuart had been appointed to design or refurbish the building services of the Queen's Gallery in Edinburgh (see Figure~\ref{fig:qg}).
Rather than taking the most energy efficient and sustainable approach to the design, he had to consider the social and commercial context of the gallery.
One of the requirements was that none of the services could be visible, as this would not look attractive to the visitors.
Another requirement was that the temperature and humidity levels needed to be maintained within strict ranges so that the artwork would not get damaged.
This is challenging considering that flows of people would be coming in and out of the gallery, possibly with their wet coats and umbrellas on rainy days, causing heat and humidity to constantly be exchanged with the outdoors and the visitors themselves.
In order to isolate the artwork from its contextual environment, so to speak (see Figure~\ref{fig:isolation}), Stuart had to have sensors installed that would detect fluctuations in temperature and humidity, as well as precise ventilation and heating systems that would constantly maintain the acceptable ranges.
This would consequently make the project more expensive, but it was considered necessary for the running of the gallery.

\begin{figure}[htbp]
	\centering
	\begin{subfigure}[b]{0.5\textwidth}
		\includegraphics[width=\textwidth]{figures/isolation.png}
		\caption[Contextual isolation.]{A graphical representation of contextual isolation that I created and used in a PowerPoint for a group debate in \ID}\label{fig:isolation}
	\end{subfigure}
	
	\begin{subfigure}[b]{.48\textwidth}
		\centering
		\includegraphics[height=5cm]{figures/gallery-exterior.jpg}
		\caption{Entrance \citep{AboutTheQueensGallery}}\label{fig:qgexterior}
	\end{subfigure}
	\begin{subfigure}[b]{.48\textwidth}
		\centering
		\includegraphics[height=5cm]{figures/gallery-interior.jpg}
		\caption{Interior \citep{InsideTheGallery}}\label{fig:qginterior}
	\end{subfigure}
	\rule{\textwidth}{0.5pt} % use line???
	\caption[The Queen's Gallery and the concept of contextual isolation.]{The Queen's Gallery and the concept of contextual isolation in the design of its building services}
	\label{fig:qg}
\end{figure}





\subsection*{EL3(i, b, m)}

\begin{wraptable}{r}{0.2\textwidth}
	\begin{tabular}{|ll|}
		\hline
		\multicolumn{2}{|c|}{\cellcolor[HTML]{F8A102}\textbf{EL3(i, b, m)} \nomaster} \\ \hline
		\DST & \LAB \\
		\ICP &  \\ \hline
	\end{tabular}
\end{wraptable}

I gained knowledge of some nascent management techniques that may be used to achieve engineering objectives.
In \ICPTitle, I learned about techniques such as lean construction and supply-chain management (which both have the aim of helping identify flows and wastes) as well as partnering and BIM (which both encourage openness and transparency).

Take lean construction, which is a concept borrowed and adapted from lean manufacturing.
This production method was invented by the Japanese company Toyota shortly after the Second World War.
Japan at the time found itself especially limited in resources due to trade embargoes or restrictions with the Western powers.
Toyota thus had to find a way to make the most use of its restricted resources in order to achieve its production objectives.
Hence, lean manufacturing and construction are about achieving engineering objectives in the most efficient way, i.e. by minimising waste and maximising value.
There are several planning systems that complement this management technique, for example Last Planner which is designed to produce predictable workflows (see Figure~\ref{fig:lastplanner}).


\begin{figure}[htbp]
	\centering
	\includegraphics[width=10cm]{figures/last-planner.jpg}
	\rule{10cm}{0.5pt} % use line???
	\caption[The Last Planner system.]{The Last Planner system \citep{last-planner}.}
	\label{fig:lastplanner}
\end{figure}


I have some understanding of management techniques since I have used Gantt-like charts on a couple occasions (see Figures~\ref{fig:lab-gantt} and~\ref{DST_schedule}).
When I used the chart to plan the execution of a \LABTitle \space acoustics experiment, it was not very successful.
The chart attempted to coordinate the overlapping activities of our group (which consisted of seven members) within a time limit of 40 minutes so that we could carry out trials on three groups of research participants.
In the end, our experiment ran over time due to our poor timing/ coordination.
The use of the chart was not successful because it was very detailed (on a minute-by-minute basis) and yet we had not familiarised ourselves with it well enough beforehand.
If we were well-rehearsed in the experimental procedure, the use of the Gantt chart could have been more successful.
Hence I learned that there is an element of practice that needs to complement management techniques in order for them to work or run smoothly.

I have not come across change management and do not have enough of experience with management techniques to know their limitations and how they may be applied appropriately.
%Judging from the titles of the upcoming courses of Semester 2, Year 5 (see Table~\ref{tbl:courses}), I suspect I might achieve LO EL3m.
%Considering that this LO should be gained at a Masters level, maybe t
Considering that LO EL3m should be gained at a Masters level, I might learn these things in the second semester of Year 5.
However, judging from the titles of the upcoming courses (see Table~\ref{tbl:courses}), this to me sounds unlikely.





\subsection*{EL4(i, -)}

\begin{wraptable}{r}{0.2\textwidth}
	\begin{tabular}{|ll|}
		\hline
		\multicolumn{2}{|c|}{\cellcolor[HTML]{F8A102}\textbf{EL4(i, -)} \master} \\ \hline
		\IE & \DPA \\
		\CAS & \EnBldgs \\
		\DI & \FMP \\
		\SIB & \CCSA \\
		\WSD &  \\
		Arup & Hoare Lea \\
		Sunamp & Sweco \\ \hline
	\end{tabular}
\end{wraptable}

I have a good understanding of the requirement for AE activities to promote sustainable development as many of the AE courses have addressed the increasing need for sustainable design to mitigate and adapt to climate change.
%According to the oft-quoted Brundtland report, sustainable development is an approach to progress which meets the needs of the present generation without compromising the ability of future generations to meet their own needs.
%In the context of the built environment, this can be translated as, ``Designing, constructing and managing buildings and resources in such a way that building occupants’ needs are met without the profligate use of energy and resources, such that sufficient provision is left for future generations to provide for themselves" \citep{CCSAunit1}.
%Perhaps because of the direct influence AE has on particularly the energy consumption of a building, many of the courses throughout the programme have addressed the increasing need for sustainable design to mitigate and adapt to climate change.
I especially realised the importance of this engineering requirement as I came across a response CIBSE offered in a governmental consultation during my research for a group poster for \CCSATitle.
%(see description of assignment in \textit{G1} in Section \ref{sec:G1}), 
%At the start of 2018,
Before the publication of the 2018 National Adaptation Programme (NAP) to climate change, 
the Department for Environment, Food \& Rural Affairs (DEFRA) suggested that adaptation reporting continues to be done on a voluntary basis.
However, CIBSE protested by stressing that adaptation is necessary because climate change threats are real and imminent, so reporting should be done on a mandatory basis by even more sectors than have been required of so far \citep{CIBSE:CCAreporting}.
Unfortunately, according to the 2018 NAP, the government will not make reporting mandatory because the majority of the respondents to the consultation favoured the continuation of voluntary reporting \citep{DEFRA2018}.
Perhaps if more engineering institutions acted like CIBSE and promoted the urgency of sustainable development in this consultation, as is their duty, the government's stance on sustainable development might have been stronger.

\begin{wrapfigure}{r}{0.5\textwidth}
	\centering
	\includegraphics[width=0.5\textwidth]{figures/SVL.PNG}
	\rule{0.5\textwidth}{0.5pt} % use line???
	\caption[The \textit{Miljöbyggnad} assessment criteria for solar heat gains in existing buildings.]{The \textit{Miljöbyggnad} assessment criteria for solar heat gains (in W/m\textsuperscript{2}) in existing buildings \citep{SwedenGreenBuildingCouncil2017}.}
	\label{fig:svl}
\end{wrapfigure}

I also have the ability to apply quantitative techniques to promote sustainable development.
An example is the work I did at Sweco on environmental certification, a way to promote the sustainable design of buildings.
I calculated and ranked things like solar heat gains, energy consumption and daylighting according to \textit{Miljöbyggnad}'s standards of what qualified as a sustainable building (see Figure~\ref{fig:svl}).


%According to the oft-quoted Brundtland report, sustainable development is an approach to progress which meets the needs of the present generation without compromising the ability of future generations to meet their own needs.
%\hl{Can't think of examples. Also, ability to apply quantitative techniques???}






\subsection*{EL5(-, m)}

\begin{wraptable}{r}{0.2\textwidth}
	\begin{tabular}{|ll|}
		\hline
		\multicolumn{2}{|c|}{\cellcolor[HTML]{F8A102}\textbf{EL5(-, m)} \master} \\ \hline
		\PC & \DI \\
		\ISE & Hoare Lea \\
		Sunamp & Sweco \\ \hline
	\end{tabular}
\end{wraptable}

A few courses have increased my awareness of legal requirements governing engineering activities including H{\&}S, liability issues, contracts and accessibility.
In particular, in \textit{Design Issues} I learned about the various UK regulations surrounding H\&S, including the Health and Safety at Work Act 1974 and the Construction (Design and Management) Regulations 2015, a.k.a. the CDM Regulations.
The CDM Regulations emphasise that H\&S need to be considered, not just on site, but throughout the planning and design stages also.
The designers, for example, have a responsibility to assess and eliminate or mitigate the H\&S risks associated with their design, whether that is during the construction or operation of the building.
Furthermore, this course broadened my awareness of legal requirements outside of the UK.
I learned about the H\&S regulations in France, the lack of building regulations in Guyana \citep{Guyana_regulations}, and the ambitious building regulations with regards to sustainability in Singapore \citep{Singapore2008}.
In France, for example, the management of H{\&}S at work is accomplished thanks to the requirement of the Unique Document of Evaluation of Professional Risks \citep{DU2009}.

I have also applied my knowledge of legal requirements in a project for \textit{Inclusive and Safe Environments} and during my placements at Hoare Lea and Sweco.
At Hoare Lea I produced RIBA Stage 3 layouts of fire detection and alarm installations for a new-build.
These were made to comply with standards such as \textit{BS 5839-6:2013 Fire detection and fire alarm systems for buildings - Part 6: Code of practice for the design, installation, commissioning and maintenance of fire detection and fire alarm systems in domestic premises}.
For instance, the standard stated the different radiuses that smoke detectors and heat detectors cover; I had to thus place the detectors on the plan layouts accordingly.







\subsection*{EL6(i, b, m)}

\begin{wraptable}{r}{0.2\textwidth}
	\begin{tabular}{|ll|}
		\hline
		\multicolumn{2}{|c|}{\cellcolor[HTML]{F8A102}\textbf{EL6(i, b, m)} \nomaster} \\ \hline
		\ConTechOne & \IE \\
		\ELS & \TPS \\
		\DI & \FMP \\
		\LAB & \ISE \\
		\SIB & \ICP \\
		\CCSA & \WSD \\
		Arup & Sunamp \\ \hline
	\end{tabular}
\end{wraptable}

Many courses and a couple of my placements increased my awareness of risk issues.
In \textit{Electrical and Lighting Services for Buildings}, for example, I learned about the importance of earthing to give overflow electricity an alternative path to run through which is not a person (thus electrocuting them).
I have also learned about environmental risk issues, notably the impact of buildings and building systems on the environment, e.g. harmful refrigerants in air conditioners, embodied carbon in building materials, and the emissions from vehicles that people use to travel to buildings.
%Lastly, I learned a little about commercial risk in \textit{Innovation in Construction Practice}, more specifically the risk of innovation in construction projects.

I learned about the techniques of risk assessment and risk management in depth in \textit{Design Issues} and have carried out H\&S risk assessments in \textit{Facilities Management Principles}, \textit{Laboratory Project} and during my placement at Arup.
Generally, one first needs to identify the risks and then assess them in terms of severity and likelihood (see Figure~\ref{fig:RA_matrix}).
Afterwards, one tries to eliminate the risks (the worst first), and if that is not possible, mitigate them.

None of my courses or work placements have provided me with an opportunity to develop the ability to evaluate commercial risk.


\begin{figure}[htbp]
	\centering
	\includegraphics[width=\textwidth]{figures/RA_matrix.png}
	\rule{\textwidth}{0.5pt} % use line???
	\caption{A risk analysis matrix.}
	\label{fig:RA_matrix}
\end{figure}






\subsection*{EL7m}

\begin{wraptable}{r}{0.2\textwidth}
	\begin{tabular}{|ll|}
		\hline
		\multicolumn{2}{|c|}{\cellcolor[HTML]{F8A102}\textbf{EL7m \littlemaster}} \\ \hline
		\ID & \PC \\
		\FMP & \ICP \\ \hline
	\end{tabular}
\end{wraptable}

For business success, I have only gained some understanding of how innovation is a key driver in \textit{Innovation in Construction Practice}.
Without innovation, you might be put of business because your competitor is being innovative.
% and \hl{how technology (?) plays a role in business success through the study of Fordist and post-Fordist consumerism/ manufacturing?} in \textit{Sustainable and Intelligent Buildings}.
%\hl{Look up ICP and SIB notes + elaborate.}

Due to the nature of my academic programme, I have not learned as much about the key drivers for business success as I have for successful construction projects.
However, there are two ways that this knowledge can translate to business.
Firstly, a construction project is often a means for a business to grow or improve (this is known as a secondary business case).
Therefore, a successful construction project can lead to a successful business.
Secondly, the principles behind a successful construction project should also be able to be applied to businesses.
For example, after the construction sector was pointed out in the 1990s for repeatedly delivering projects that were of poor quality, overtime and over-budget, the sector has made significant efforts to improve its performance and improve client satisfaction.
Such efforts have resulted in changing the workflow so that more planning is done at the start of a project, when making changes is more flexible and less costly (see Figure~\ref{fig:cost_V_time}).
This includes developing more descriptive briefs, which set out the client's needs and how the construction project should be run.
Translated to business, configuring a thorough strategy at the start of a project will help lead to business success.


\begin{figure}[htbp]
	\centering
	\includegraphics[width=0.5\textwidth]{figures/cost_V_time.png}
	\rule{\textwidth}{0.5pt} % use line???
	\caption[The cost of changes in construction projects.]{The cost of changes in construction projects. N.B. O{\&}M stands for operation and maintenance.}
	\label{fig:cost_V_time}
\end{figure}
%----------------------------------------------------------------------------------------
%	SECTION 5
%----------------------------------------------------------------------------------------

\section{Engineering Practice (P)}


\subsection*{P1(i, -)}

Throughout the programme, I have gained a knowledge and understanding of some contexts in which my Architectural Engineering knowledge can be applied.
The following list in which I describe such contexts is not exhaustive.

\begin{enumerate}
	\item The collaborative design projects in years \hl{1/2} and 4,
	\textit{Building Services Technology},
	\textit{Introduction to the Environment},
	\textit{Design Project A},
	\textit{Design Project B},
	\textit{Electrical and Lighting Services for Buildings},
	\textit{Critical Architectural Studies},
	\textit{Design Project},  
	and my placements at Arup and Hoare Lea
	provided me with the practical experience of applying my AE knowledge in the context of a team designing a building and/ or its services, in which I was responsible for the building services, the building's internal environment and the occupants' comfort levels.
	
	\item Through
	\textit{Acoustics and Architectural Design},
	\textit{Environment and Behaviour},
	\textit{Energy Principles and Applications},
	\textit{Energy and Buildings},
	\textit{Thermal Performance Studies},
	\textit{Design Issues},
	\textit{Facilities Management Principles},
	\textit{Laboratory Project},
	\textit{Sustainable and Intelligent Buildings},
	\textit{Climate Change, Sustainability and Adaptation},
	\textit{Water Supply and Drainage for Buildings},
	my work in the environmental certification of existing buildings at Sweco,
	and
	my encounter with a specialist in Performance at Hoare Lea,
	Design Issues,
	I have also learned that my AE knowledge is applicable in the operation and management of buildings and building services (a.k.a. facilities management) to, for example, improve the occupants' comfort levels, optimise performance or increase resilience to future impacts of climate change.
	
	\item Through the work I did at Sunamp as well as my encounter with a technical author at Hoare Lea, I understand that engineering knowledge is important in the composition of technical documents (e.g. specifications, operation and maintenance manuals) and even marketing material for technical products, which may need to explain engineering processes in layman's terms.
	
	\item My \textit{Dissertation} in particular taught me that engineering knowledge is necessary for the development of industry standards and codes of practice, such as the series of Publicly Available Specifications (PAS) 1192 which standardise the requirements for achieving BIM Level 2.
	
	\item Throughout my \textit{Dissertation} and \textit{Climate Change, Sustainability and Adaptation}, I have also come to understand that my engineering knowledge can be used for research, the development of new technologies or processes, and to influence policies. \hl{Repeated DST and CCSA. Could even include Sunamp. Maybe this example is one too many...}
\end{enumerate}


\subsection*{P2(i, b, m)}

I have gained a knowledge of the characteristics of, an understanding of and an ability to use a range of computer-based tools, building services products, and engineering processes:
\begin{itemize}
    \item P2i 
     
     In \textit{Environment and Behaviour}, \textit{Laboratory Project} and \textit{Dissertation}, I developed an ability to use the statistics software tool SPSS (Statistical Package for the Social Science).
     The course \textit{Statistics for Science} provided me with a foundation to understand the data in the inputs and outputs of this software.
    
    \item P2i and P2
    
    In \textit{Design Software Applications} and \textit{Laboratory Project}, I gained an in-depth understanding of the characteristics of steady-state and dynamic building modelling and energy analysis software programmes, notably iSBEM, SAP, IES, and CFD.
    As these courses only provided an introduction to these software programmes, I have not mastered my ability to use them.
    This was demonstrated during my attempt to model my \textit{Design Project} building in IES; my building model overheated due to variation and temperature profiles that I had incorrectly set up, amongst other things.
    
    During the \textit{Design Project} and my placements at Hultin \& Lundquist Arkitekter and Sunamp, I gained an understanding of and an ability to use Autocad.
    Likewise, I learned to use Revit during the \textit{Design Project} and my placements at Arup and Hoare Lea.
    Although my ability to use these Autodesk products (i.e. Autocad and Revit) is limited, I have learned quite a bit about their characteristics.
    In \textit{Dissertation} and \textit{Innovation in Construction Practice}, I learned about the prominent use of these products in the UK construction industry and how their proprietary characteristics can lead to problems of interoperability.
    
    Regarding engineering processes, I have learned extensively about the BIM process in Dissertation and Innovation in Construction Practice.
    I have also learned the characteristics of and practised the optimisation process of an engineering design in \textit{Laboratory Project} (see EA3(i, b, m) in Section \ref{EA} for more detail \hl{is this the correct reference where I expand on the optimisation of cooling coil?}).
    
    \item P2m
    
    I have gathered an extensive knowledge and understanding of a wide range of building services products (e.g. heat pumps, heat batteries, photovoltaic panels) and building materials (e.g. phase-change materials, Ziegel blocks, steel)
    throughout many of my construction-based courses, i.e. 
    \textit{Construction Technology 1},
    \textit{Building Services Technology},
    \textit{Design Project A},
    \textit{Construction Technology 2},
    \textit{Acoustics and Architectural Design},
    \textit{Design Project B},
    \textit{Critical Architectural Studies},
    \textit{Electrical and Lighting Services for Buildings},
    \textit{Design Software Applications},
    \textit{Energy and Buildings},
    \textit{Thermal Performance Studies},
    \textit{Design Project},
    \textit{Laboratory Project},
    \textit{Sustainable and Intelligent Buildings},
    \textit{Innovation in Construction Practice},
    and \textit{Water Supply and Drainage for Buildings},
    and my placements at Sunamp, Hoare Lea and Arup
\end{itemize}
\hl{Write about increase in knowledge throughout the years. P2m for example couldn't all of a sudden have been achieved in Y5.}

%Building materials: insulation, steel, wood, Ziegel blocks, glass, PCMs etc.

%Building services products: ACs, HPs, PV, AHUs, Sunamp, windcatchers


\subsection*{P3(i, -)}

\hl{I have had laboratory practice in the realms of physics, chemistry and biology in high school (mention?).}
In university, my knowledge and understanding of laboratory practice began to further develop in Year 2 when I conducted experiments in \textit{Construction Technology 2}, \textit{Acoustics and Architectural Design}, \textit{Hydraulics and Hydrology A} and \textit{Energy Principles and Applications}.
The feedback for these lab reports (\hl{respectively ..., 93\% and ...}) demonstrate my knowledge and understanding of laboratory practice.

In Years 3 and 4, I gained the ability to apply practical and laboratory skills that were more relevant to Architectural Engineering through my increased exposure to and application of laboratory practice in thermofluids and acoustics.
I re-visited the Services lab another two times for thermofluids-related experiments in \textit{Thermal Performance Studies} and \textit{Laboratory Project} and re-used my practical skills of recording, measuring and analysing sounds in \textit{Critical Architectural Studies} and \textit{Laboratory Project}.
\hl{What actions and results can I show as evidence of my practical skills?}


\subsection*{P4(i, -)}

The technical literature I have been exposed to includes Sunamp's manuals for their UniQ product range and the product information sheets I drew up for them.
These sheets included information such as temperature input and output ranges, wiring diagrams, and the dimensions of the heat batteries.
\hl{Include evidence, e.g. attach PISs}

I also came across and used technical literature during my \textit{Design Project}.
I used product specifications to size the college's water storage tank \citep{Decca}, calorifiers and buffer vessels \citep{RycroftLtd} etc.

Throughout my \textit{Design Project} and Sunamp experiences, I gained an understanding of the use of technical literature.
During the Design Project, I used technical literature to appropriately size some of the building services, both to accommodate the needs of the building's occupants and to fit inside of the plant room.
At Sunamp, I was part of the decision-making process of the information that should be included in the product information sheets, what would be one of the first documents a customer would see once they have taken interest in Sunamp's products.
It was decided that information such as dimensions and weight were important to, firstly, highlight the compact size of Sunamp's batteries (which is one of their unique selling points) and, secondly, give the customers an indication of the transportation and placement requirements of a battery (e.g. it may need to be carried by two people and it needs to be placed on a load-bearing floor).
\hl{Perhaps do not be afraid to write more sentences and give details such as specific weights and dimensions, and talk about stacking batteries and load bearing floors vs shelves etc.}


\subsection*{P5}

Considering that in \hl{AHEP} defines ``knowledge" as information that can be recalled, I do not have much knowledge of legal and contractual issues relevant to Architectural Engineering.
We have, however, covered \hl{contracts...} in \textit{Procurement and Contracts} and professional liability and insurance in \textit{Design Issues}.
It is perhaps due to a lack of coursework or examination and/ or a lack of spaced repetition of these topics throughout my degree programme that have I not gained a firm knowledge in relevant legal and contractual issues.


\subsection*{P6(i, -)}

I have used appropriate codes of practice and industry standards during \textit{Design Project}, \textit{Inclusive and Safe Environments} and my placements at Hoare Lea and Sweco.
For the \textit{Design Project}, for example, I wanted to use water source heat pumps (WSHPs) to generate heat for the college since it was located right next to a body of water \hl{(see Figure ...).}
I had to figure out how WSHPs could fit into the building's heating strategy.
To do this, I read about WHSPs in CIBSE CP2, the UK's Code of Practice for surface WSHPs.
I decided, because the college was a large non-domestic facility, to use a multi-valent heating system, where the WSHPs would provide the base heating load and some micro combined heat and power (CHP) units would satisfy the peak heating demands \citep[pp.~12,~38]{CP22016}.
\hl{I have no feedback to demonstrate this specifically was good and my grade for DP wasn't good because of missing elements...}


\subsection*{P7}


\subsection*{P8}

I think part of \hl{my failure?} of the Design Project was my inability to work with technical uncertainty.
I have a tendency to be detail-oriented and thus get stuck in details etc., to not take risks and to be indecisive
But I did (painstakingly) work through some uncertainties, thus achieving a passing grade (?).


\subsection*{P9m}


\subsection*{P10m}


\subsection*{P11(i, b, m)}



%----------------------------------------------------------------------------------------
%	SECTION 6
%----------------------------------------------------------------------------------------

\section{Additional General Skills (G)}


\subsection*{G1}

% Please add the following required packages to your document preamble:
% \usepackage[table,xcdraw]{xcolor}
% If you use beamer only pass "xcolor=table" option, i.e. \documentclass[xcolor=table]{beamer}
\begin{table}[h]
	\begin{tabular}{|c|}
		\hline
		\rowcolor[HTML]{F8A102} 
		\textbf{G1} \\ \hline
		All courses and \\
		placements \\ \hline
	\end{tabular}
\end{table}

I have had several opportunities to apply my skills in problem solving, communication, information retrieval, working with others and the effective use of general IT facilities.
I will demonstrate my application of these skills by providing examples from the work I produced for Sunamp and \textit{Climate Change, Sustainability and Adaptation} (D11CA).
%A good example is when I applied these skills to produce the UniQ Overview Sheet, Product Selection Quiz and PISs for Sunamp (see \Cref{App:Overview,App:quiz,App:PISs}).
I applied my skills in:
\begin{itemize}
    \item Problem solving at Sunamp, as demonstrated in Section \ref{sec:quiz}, by solving the problem of guiding customers to their most suitable UniQ products through the creation of a quiz (see Appendix \ref{App:quiz}).
    %when I encountered inconsistencies or unclear information in the UniQ manuals.
    %For example, I was tasked with providing accurate and up-to-date dimensions of the UniQ batteries because the dimensions provided in the UniQ manuals were inconsistent.
    %Therefore, I asked a staff member what might be the cause for the discrepancy and they explained that the different dimensions might be referring to different UniQ prototypes.
    %They also suggested that I measure the batteries that were currently being mass-produced.
    %So I asked for a tape measure and used it to measure the dimensions of the most current UniQ batteries.
    %I also cross-checked my measurements with the staff member's until we settled on reasonable nominal dimensions.
    %As a result, I created a diagram with the nominal dimensions... 
    
    \item Communication during a recent D11CA assignment in which the aim was to write a POSTnote.
    POSTnotes are briefings on public policy issues that are based on academic and industrial research.
    They are written to inform the Parliamentary Office of Science and Technology (POST) so that appropriate policies can be developed.
	The crux of the assignment was to write technical information in layman's terms so that any politician could fully understand the issue at hand before entering a parliamentary debate or meeting.
	I wrote my POSTnote on adapting UK households to a 2050 climate and received a provisional mark of 67\%.
	My communication was successful, as my feedback reflects:
	``Overall, this is very clearly written and easy to understand."
    %as I produced the documents.
    %With the development of the quiz, I managed to come up with a way to efficiently direct customers to the most suitable UniQ product for them, which made the customers more informed and reduced the workload for the sales team.
    
    \item Information retrieval at Sunamp in order to produce accurate PISs (see Appendix \ref{App:PISs}).
    This was done by reading the UniQ manuals and other documents, doing online research and asking staff members for explanations to fill my gaps in understanding, and hands-on experience (e.g. measuring the dimensions of UniQ batteries).
    
    \item Working with others at Sunamp in order to produce the UniQ documents, which was a collaborative effort.
    (See Sections \ref{sec:quiz} and \ref{sec:piss} for references to collaborations and iterative feedback processes.)
    
    \item The effective use of general IT facilities at Sunamp throughout the production of the UniQ documents.
    For example, I would regularly print out my latest drafts for easy marking-up during meetings and back up my drafts online on Google Drive.
    I also applied this skill in D11CA by converting a Word document (docx) to a Portable Document Format (pdf) so that the POSTnote could be displayed consistently over different IT systems.
    %utilised media conversion techniques to ensure that the same message was displayed consistently over different IT systems
    % I created the Overview Sheet on Microsoft Publisher, the quiz on Microsoft PowerPoint, and the PISs on Microsoft Word. % MS applications are just software, not facilities.
\end{itemize}


\subsection*{G2}

DST
Work placement planning
Improve performance?


\subsection*{G3(i, b, m)}

\hlc[green]{On reflection, I should have created the same Gantt chart to plan my work on this IP report. Now I feel like I have a whole lot to do and very little time.}

Throughout my years at Heriot-Watt University, I have increasingly planned and carried out a personal programme of work.
This culminated in Year 4, when we had the \textit{Design Project} and \textit{Dissertation} which were both long-term personal projects.
In the interim report that I wrote for \textit{Dissertation}, I included a personal programme of work (similar to a Gantt chart) that mapped out when and how I would work on the different sections of my dissertation for the rest of the academic year (see Figure \ref{DST_schedule01}).
I adjusted the programme as I carried it out (see final programme in Figure \ref{DST_schedule02}).
The major variations of the last programme from the first are:
\begin{itemize}
    \item No more work was done in Semester 1 and during the Christmas holidays. This was because I had decided that it was better for me to use that time to focus on my Design Project instead.
    \item The programme was more compressed, due to the above reasons.
    \item The blue block was segmented into two blocks. Blue represented my work on questionnaires and interviews. The gap was introduced to allow the respondents time to answer my survey before I interviewed them.
\end{itemize}

\hl{Whereas I mostly assisted my supervisors with their projects at my other placements, the work I did at Sunamp was very much a personal project.
...}


\begin{figure}[htbp]
    \centering
        \begin{subfigure}{.48\textwidth}
          \centering
          \includegraphics[width=\textwidth]{figures/DST-schedule-start.PNG}
%          \rule{\textwidth}{0.5pt} % use line???
          \caption{Initial programme}
          \label{DST_schedule01}
        \end{subfigure}
        \begin{subfigure}{.485\textwidth}
          \centering
          \includegraphics[width=\textwidth]{figures/DST-schedule-end-big.PNG}
%          \rule{\textwidth}{0.5pt} % use line???
          \caption{Final programme}
          \label{DST_schedule02}
        \end{subfigure}
    \rule{\textwidth}{0.5pt} % use line???
    \caption[My personal programme of work for my dissertation.]{My personal programme of work for my dissertation, from the end of Semester 1 to to the end of Semester 2 of Year 4.}
    \label{DST_schedule}
\end{figure}


In Year 4 I also started to monitor the hours I worked in order to adjust and improve my personal programme on an on-going basis.
Figure \ref{fig_timelog} provides an example of my weekly time log from Year 4.
By monitoring the hours I worked, I noticed a trend: I typically worked fewer hours on Thursdays than any other day.
I proceeded to identify the reason for this, which is that I tended to feel tired or burnt out by that day of the week.
After this discovery, I allocated myself less hours to work and focused on less demanding tasks (if possible) on Thursdays.
The hours that I lost on Thursday I would then try to compensate for throughout the rest of the week.


\begin{figure}[htbp]
	\centering
	\includegraphics[width=10cm]{figures/y4s1w6hours.PNG}
	\rule{\textwidth}{0.5pt} % use line???
	\caption{Time log from Week 6 of Semester 1, Year 4.}
	\label{fig_timelog}
\end{figure}


\subsection*{G4(i, -)}

My degree programme at Heriot-Watt University and my work placements have provided me with multiple opportunities to work as part of a team.
\hl{List courses/ projects.}
In all of the projects, I have exercised personal responsibility, and sometimes also initiative, as a team member.
For example, the work I produced for Sunamp, including the quiz I came up with (see Section \ref{sec:sunamp_work}), is a display of me exercising initiative and my personal responsibility to the company.

In most academic projects, however, I have exercised personal responsibility, and sometimes initiative, as a team leader.
% Sometimes this has involved me exercising initiative also.
A recent example of this is a group project of creating and presenting a poster for \textit{Climate Change, Sustainability and Adaptation}.
I assumed a leading role in this group mainly by coordinating the group activity (e.g. arranging and leading meetings, setting up weekly agendas).
For this project, I sketched the drawing in Figure... \hl{Bring initial sketch to campus and scan} and suggested the use of such an image to graphically show the importance of our topic.
After the group agreed on its usefulness to the poster, I went ahead and developed a final version (see Figure ...).
This was an act of initiative because the use of a representative image had previously not been discussed within the group.


\begin{figure}[htbp]
    \centering
        \begin{subfigure}{.48\textwidth}
          \centering
          \includegraphics[width=\textwidth]{figures/J+S_Meetings.png}
%          \rule{\textwidth}{0.5pt} % use line???
          \caption{\hl{Initial sketch}}
          \label{Sketch01}
        \end{subfigure}
        \begin{subfigure}{.485\textwidth}
          \centering
          \includegraphics[width=\textwidth]{figures/eposter_sketch_2.png}
%          \rule{\textwidth}{0.5pt} % use line???
          \caption{Final image}
          \label{Sketch02}
        \end{subfigure}
    \rule{\textwidth}{0.5pt} % use line???
    \caption{The representative graphic I suggested and developed to use on a group poster to show the importance of adapting UK houses to future climate impacts.}
    \label{fig:Sketch}
\end{figure}
%----------------------------------------------------------------------------------------
%	SECTION 7
%----------------------------------------------------------------------------------------

\section{Skills Analysis Conclusion} \label{sec:skills_ccl}


\begin{wrapfigure}[6]{r}{0.25\textwidth}
	\centering
	\includegraphics[width=0.25\textwidth]{figures/LO_pie_3.png}
	\rule{0.25\textwidth}{0.5pt} % use line???
	\caption{My achievement of the 42 LOs.}
	\label{fig:LO_pie}
\end{wrapfigure}

Out of the 42 LOs, I have reached full achievement of 11 (26\%), above average achievement of 20 (48\%), and limited achievement of 11 (26\%) (see Figure~\ref{fig:LO_pie}).
Table~\ref{tbl:LOs_summary} summarises my overall achievement levels for all of the LOs.


\begin{table}[htbp]
	\caption{My overall achievement level for every LO.}
	\label{tbl:LOs_summary}
	\centering
	\begin{tabular}{@{}cccc@{}}
		\toprule
		\textbf{} & \littlemaster & \nomaster & \master \\ \midrule
		\multirow{4}{*}{\textbf{SM}} & SM5m & SM1(i, b, m) & SM4m \\
		&  & SM2(i, b, m) &  \\
		&  & SM3(b, m) &  \\
		&  & SM6m &  \\ \midrule
		\multirow{4}{*}{\textbf{EA}} & EA6m & EA1(i, b, m) & EA5m \\
		&  & EA2(i, -) &  \\
		&  & EA3(i, b, m) &  \\
		&  & EA4(i, b, m) &  \\ \midrule
		\multirow{4}{*}{\textbf{D}} & D3(i, b, m) & D4(i, -) & D1(i, -) \\
		&  & D5(i, -) & D2(i, -) \\
		&  & D7m & D6 \\
		&  & D8m &  \\ \midrule
		\multirow{3}{*}{\textbf{EL}} & EL3(i, b, m) & EL1(-, m) & EL4(i, -) \\
		& EL7m & EL2 & EL5(-, m) \\
		&  & EL6(i, b, m) &  \\ \midrule
		\multirow{6}{*}{\textbf{P}} & P4(i, -) & P1(i, -) & P3(i, -) \\
		& P5 & P2(i, b, m) & P7 \\
		& P6(i, -) & P11(i, b, m) &  \\
		& P8 &  &  \\
		& P9m &  &  \\
		& P10(i, b, m) &  &  \\ \midrule
		\multirow{2}{*}{\textbf{G}} &  & G2 & G1 \\
		&  & G3(i, b, m) & G4(i, -) \\ \bottomrule
	\end{tabular}
\end{table}


Interestingly, over the years, my courses and placements have increasingly contributed to my development of the Engineering Council's LOs (see Figure~\ref{fig:cum_courses} in Appendix~\ref{App:matrix}).
In particular, the \LABTitle \space contributed to almost every LO.
The `runners-up' of courses/ placements that have contributed the most to the LOs are \CASTitle, \TPSTitle, \PRJTitle, and Sunamp.
It is surprising that the latter two are runners-up considering my below par performance on the \PRJTitle \space and my very non-technical work experience at Sunamp.






\subsection*{Fully Achieved LOs}

I have fully achieved between one and three LOs from every category.
These have to do with my:
\begin{itemize}
	\item Awareness of and ability to investigate emerging technologies [SM4m, EA5m]
	\item Ability to understand and evaluate user needs [D1(i, -)]
	\item Ability to define and investigate problems [D2(i, -)]
	\item Ability to communicate to technical and non-technical audiences [D6]
	\item Understanding of the need for sustainable development [EL4(i, -)]
	\item Awareness of legal requirements governing engineering activities [EL5(-, m)]
	\item Laboratory skills [P3(i, -)]
	\item Awareness of quality issues and their application to continuous improvement [P7]
	\item Ability to apply additional general skills, e.g. effectively use general IT facilities and exercise personal responsibility in a team [G1, G4(i, -)]
\end{itemize}

Most of the LOs I have fully achieved are related to things I am highly interested in, notably meeting a client's needs, the need for monitoring and evaluation in continuous improvement, and the benefits of developing technologies, all in the interest of sustainable development.




\subsection*{Limited Achievement of LOs}

However, this analysis has highlighted several skills in which I am deficient and need to work harder on to develop.
According to Table~\ref{tbl:LOs_summary}, these have to do with my limited:
\begin{itemize}
	\item Understanding of mathematical and computational models [SM5m]
	\item Ability to work with unfamiliar, uncertain or incomplete information [EA6m, D3(i, b, m), P8]
	\item Knowledge of management techniques [EL3(i, b, m)]
	\item Limited understanding of key drivers for business success [EL7m]
	\item Ability to use technical literature, codes of practice and industry standards [P4(i, -), P6(i, -)]
	\item Knowledge of contractual and legal issues [P5]
	\item Understanding of current practice [P9m]
	\item Ability to apply engineering techniques while considering commercial and industrial constraints [P10m]
\end{itemize}

Additionally, Figure~\ref{fig:cum_LOs} in Appendix~\ref{App:matrix} shows two LO attainment levels that I have no experience in:
\begin{itemize}
	\item EA3m: Ability to apply quantitative and computational methods, using alternative approaches and understanding their limitations, in order to solve engineering problems and implement appropriate action
	\item D4: Apply advanced problem-solving skills, technical knowledge and understanding, to establish rigorous and creative solutions that are fit for purpose for all aspects of the problem including production, operation, maintenance and disposal
\end{itemize}


Table~\ref{tbl:LOs_summary} shows that 55\% of the LOs I have limited achievement in are under the Engineering Practice (P) category.
I believe I should be able to strengthen these skills once I start to work in industry.
Another overarching theme is my difficulty working with uncertain information.
This is most likely linked to my perfectionist and risk-averse tendencies.
To overcome this, I need to practise using my best judgment, making quicker decisions and generally taking more risks.

