\chapter{Conclusion} % Main chapter title

\label{Chapter11} % Change X to a consecutive number; for referencing this chapter elsewhere, use \ref{ChapterX}

\lhead{Chapter 11. \emph{Conclusion}} % Change X to a consecutive number; this is for the header on each page - perhaps a shortened title

% The aim of this research study is to investigate the discrepancies between industry guidance and practice in building services engineers' processes for collaboration in the UK.
% To provide context before the discrepancies are sought, the concept of collaboration will be explored, along with the collaboration-enabling tools on the market for building services engineers.

This study revealed five discrepancies between industry guidance and practice in building services engineers' processes for collaboration in the UK.
The first is the government's request for the widespread adoption of BIM since 2016, yet there is still a wide range of BIM awareness and knowledge in the industry.
The second is the BG 6's recommendation of substituting generic BIM objects with specific BIM objects during equipment procurement, but this is technically not possible at the moment.
Instead, the lack of BIM interoperability is engendering a time-consuming process of re-work. %, the value of which is arguably wasteful or beneficial in the long-term.
The third is the government's request for early contractor involvement and the use of proprietary BIM objects from the outset of a project, which requires a practical knowledge and skill set.
This conflicts with the purely theoretical knowledge and skill set of most consultants, if the consultants are the ones to be designing the building services at the start of a project.
The fourth is the industry's continued use of proprietary file formats (e.g. maj) in spite of the government's request to deliver projects to BIM Level 2, which is distinguished by data exchange through common file formats (e.g. IFC).
The fifth (and last) is the industry's introduction of LODs to facilitate communication among BIM users, yet in reality, the LOD terms seem to have caused more confusion.

The implications of these discrepancies are varied.
The contractor's re-work engendered from BIM interoperability is arguably inefficient or valuable for making a smoother transition to construction and installation.
The government's construction strategies suggest an imminent social reformation of the UK's AEC industry, which may disgruntle certain professionals who consider themselves high up in the current hierarchy.
% request for early contractor involvement among other things is a signs of a potential upcoming reformation of the UK AEC industry
The abortive introduction of the LOD terms might instigate the creation of a more effective communication system for BIM.
With BIM being a recent innovation in both technology and process, its implementation in the industry is itself a sort of design exercise.
It will require iterations of trial and error before (a) the industry gets accustomed to BIM process and technology and (b) the industry is able to adapt BIM technology and process to suit its needs.
% The implementation of the BIM process is a design exercise in itself, with iterations to see which processes work and which do not…
As \cite{Miettinen2014} suggest, this `design exercise' may take up to two decades.
% Practice makes perfect?

%----------------------------------------------------------------------------------------
%	SECTION 1
%----------------------------------------------------------------------------------------

\section{Limitations of Current Research}

Limitations in this study included time and resources.
These could have contributed to getting a larger sample size for the survey.
If a greater sample size that is less skewed and more representative of the UK population of building services engineers could have been acquired, more advanced and accurate statistical analysis could have been conducted.
% Such statistical analysis include 
% If the author and the supervisor could have reached out to a greater number of practising AEC professionals, perhaps 
Also, with more time, it would have been interesting to follow-up more responses from the survey, for example to acquire more examples on how LODs are project-related and not stage-related.

% Although many in industry claimed/ expressed that LODs are project related, not work stage related, I have limited examples to demonstrate this.

% Somehow include flow charts drawn in pencil of stage to stage interactions within various disciplines which confirms info exchanges…

%----------------------------------------------------------------------------------------
%	SECTION 2
%----------------------------------------------------------------------------------------

\section{Suggestions for Further Research}

It might be interesting to investigate the information flow patterns between MEP consultants and contractors in order to identify characteristics that would facilitate the development of more effective CSCW tools.
A variant of this suggestion would be to examine the implications of the technology that enables intelligent object substitutions in BIM models, if or when this technology emerges.
It would be interesting to compare the consultant-to-contractor handover process with and without that technology (e.g. the extent of re-work on the BIM model, the number of hours worked by the contractors, and the smoothness of the transition to construction and installation on site).
Such a study would contribute to streamlining the handover process. % smoother and more efficient.

Other suggestions for further research that have resulted from the discoveries and limitations of this study are:
\begin{itemize}
	\item Further research into the development, evolution and effectiveness of LODs in the UK's AEC industry.
	This could involve investigating the sources of confusion with LODs, whether their meaning has evolved from being stage- to project-related, and exploring other means of communicating the amounts of graphical and non-graphical content in BIM models.
	The latter could include identifying alternative approaches that are currently being employed in the industry and assessing their efficiency compared to LODs.
	% Explore other means of communicating amount of model detail and information, e.g. by seeing alternative approaches currently being employed in industry (are these more efficient?).
	% More recent publications by BSI, BSRIA and NBS might confirm this speculation (This may suggest that the industry realised in 2016 that LODs should not be work stage related.)
	% LODs mistaken for coordination - how exactly?
	% Investigate sources of confusion of or problems with LODs.
	
	\item ``Does the line between building services consultants and contractors need to blur for improved productivity?" The `blurring line' could manifest itself through MEP consultants gaining more practical knowledge and/ or an increased collaboration between consultants and contractors.
	Such a study could explore the implications related to productivity, social hierarchy, technology and education.
	
	\item ``To what extent have building services engineers achieved BIM Level 2 in the UK?" As BIM Level 2 is distinguished by data exchange through common file formats, this study would involve examining how practising building services engineers are exchanging design information.
\end{itemize}

% Big range of BIM awareness discovered - further study extent of BIM technology and BIM process knowledge in industry?


% stakeholders on a construction project.

\begin{comment}
TEST: if/ when the technology emerges to make intelligent object substitutions in BIM models, compare the
(1) productivity of construction projects/ 
(2) efficiency of handover (e.g. compare number of hours worked)/ 
(3) how well the contractor works on site in the following scenarios:
\begin{itemize}
	\item The contractor simply substitutes the generic BIM objects with specific ones without doing any form of re-work
	\item The contractor re-builds the model (like they do today), learning the design intimately
\end{itemize}
\end{comment}

% How information is lost during handovers?
